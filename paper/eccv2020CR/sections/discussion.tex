In summary, we demonstrate a method to greatly improve depth estimates from
monocular depth estimators by correcting the scale ambiguity errors inherent
with such techniques. Our approach produces depth maps with accurate absolute
depth, and helps MDE neural networks generalize across scene
types, including on data captured with our hardware prototype.  Moreover, we
require only minimal additional sensing hardware; we show that a single
measurement histogram from a diffused SPAD sensor contains enough information
about global scene geometry to correct errors in monocular depth estimates.

The performance of our method is highly dependent on the accuracy of the initial
depth map of the MDE algorithm. Our results demonstrate that when the MDE
technique produces a depth map with good ordinal accuracy, where the ordering of
object depths is roughly correct, the depth estimate can be corrected to produce
accurate absolute depth. However, if the ordering of the initial depths is not
correct, these errors will not be corrected by our histogram matching procedure
and may propagate to the final output depth map.

In the optically diffused configuration, the laser
power is spread out over the entire scene. Accordingly, for distant scene points
very little light may return to the SPAD, \edit{ making reconstruction difficult (an analogous
  problem occurs with dark objects).} \edit{Thus, our method is best suited
  to short to medium-range scenes.}
\edit{CONSIDER MOVE THIS ENTIRE CALCULATION TO
SUPPLEMENT} For example, assuming an indoor scene with
fluorescent bulbs and an ambient spectral irradiance of $I_A =
1~\text{mW}/\text{m}^2$ (across the 1 nm pass band of a spectral filter matched
to the laser), we find that the laser power required to achieve a minimum SBR of
5 for a diffuse scene at $R = 3$ m and a field of view of
$\theta = 40^\circ$\textdegree can be calculated as
%
\begin{equation}
  P_{\text{min}} = I_A \cdot 4 R^2 \tan^2(\theta/2) \cdot SBR_{\text{min}},
\end{equation}
giving $P_{\text{min}}=21$ mW. We note that this is significantly
less than the 60 mW used by the Kinect sensor to diffusely illuminate a scene.

Under these scene parameters, we can compute the total incident
flux on the detector per second (derived in~\cite{mcmanamon2012ladar}) as
\begin{equation}
  P_R = P_T \cdot \rho \cdot \frac{A_{rec}}{\pi R^2} \cdot \eta
\end{equation}
where $P_T$ is the illumination power in the visible region, $\rho$ is its
albedo, $A_{rec}$ is the area of the detector region, $R$ is the distance to the
object, and $\eta$ is the quantum efficiency of the detector. For a range of
values similar to those of our experiments (detailed in the supplement), we find
that the ratio of photon detections to illumination pulses is roughly 4\%, or
well within low-flux regime where SPAD pileup is negligible~\cite{oconnortcspc}.
We also empirically validate that our method works with a diffused setup and
an output power of approximately 25 mW without significant pileup effects as
shown in Figure~\ref{fig:captured_diffuse_small} and detailed in the supplement.

\edit{We acknowledge that pileup in bright environments will ultimately limit
  the maximum range of our method. However, even when operating in the high-flux
  regime, pileup can be mitigated by reducing the amount of incident light, for
  example by reducing the aperture size or using neutral density filters, or by
  using pileup correction~\cite{Heide:2018,Rapp:2019}. Another approach to
  mitigate pileup involves taking two transient measurements, one with and one
  without laser illumination, and using their difference to approximate the
  backgrond-free transient.}



  %Note
  % that this calculation involves only on the signal-to-background photon ratio,
  % which is independent of the actual number of counts recorded per exposure. In
  % other words, pileup and laser energy concerns may 
  % be addressed entirely independently. As for acquisition time, at a 10 MHz
  % pulse rate with
  % a 5\% photon detection rate, it takes just two seconds to acquire a million
  % photon events.} 

% Furthermore, our system uses a SPAD and is therefore susceptible to pileup.
% Because the SPAD must be quenched before it can
% detect another photon, earlier photons
% from closer objects may systematically prevent detection of photons from further
% objects, causing strong distortion of the transient. Fortunately, at the expense of
% longer acquisition times, pileup can be mitigated
% to a large extent by operating in the low-flux regime
% where the number of photon events per second is controlled to be at most 5\% the
% number of laser pulses per second~\cite{oconnortcspc}. Finally, even in
% the high-flux regime, it is sometimes possible to correct pileup computationally~\cite{Heide:2018}.}

% While our prototype shows results for scanned/summed measurements, our scanning prototype operates in the low-flux regime,
% avoiding pileup. A hypothetical diffused SPAD prototype could also operate in
% this low-flux regime, which would yield identical measurement models and identical
% results. Figure~\ref{fig:hardware}(b)-(c) empirically shows that collecting and summing scanned
% measurements is nearly identical to capturing diffused measurements. ?!We show
% empirical comparisons between the two modes on a more complex scene in the supplement?!}

% Even if diffused and scanned/summed measurements are equivalent
%   computationally, they are not equivalent energetically.  This number is dependent on scene albedo, but all time-of-flight imaging
% systems share this dependency.}


%%%%%%%%%%%%%%%%%%%%%%%%%%%%%%%%%%%%%%%%%%%%%%%%%%%%%%%%%%%%%%%%%%%%%%%%%%%%%%%%%%%%
\paragraph{Future Work}
\edit{DELETE While our hardware prototype is large, future work could miniaturize this
system.} Our algorithm or similar sensor fusion algorithms could
\edit{DELETE also} be integrated into electronics that already contain the required hardware
components, for example, existing cell phones with single-pixel SPAD
proximity sensors and RGB cameras \edit{The small baseline of such sensors
  would also mitigate the effects of shading and complex BRDFs. Necessary
  adjustments, such as pairing near-infrared (NIR) SPADs with NIR sensors,
  could be made. More sophisticated intrinsic imaging techniques could also be
  employed, replacing the primitive RGB channel method described here.}


Other methods for extracting scene information from a transient could be
employed, including learning-based methods to combine the MDE estimates and
histogram. One might even consider sensing regimes where the number of
returning signal photons is low, such as when the time-resolved detector and
camera operate at high framerates. While most MDE techniques are tailored to
clean RGB images, the transient could be used to help MDE techniques generalize
to noisy scenes under low-light conditions.


%%%%%%%%%%%%%%%%%%%%%%%%%%%%%%%%%%%%%%%%%%%%%%%%%%%%%%%%%%%%%%%%%%%%%%%%%%%%%%%%%%%%
\paragraph{Conclusions}
Since their introduction, monocular depth estimation algorithms have improved tremendously.  However, recent advances, which have generally relied on new network architectures or revised training procedures, have produced only modest performance improvements. 
In this work we dramatically improve the performance of several monocular depth estimation algorithms by fusing their estimates with transient measurements. 
Such histograms are easy to capture using time-resolved single-photon detectors and are poised to become an important component of future low-cost imaging systems.
%Moving forward, sensor fusion is poised to become an even more important component of a variety of vision systems.
%Computational photography approaches may provide the measurements needed to achieve much greater performance improvements, especially as the sensors in devices we use every day become more diverse, complex, and increasingly rely on techniques based on sensor fusion and sophisticated computational algorithms.
 
%\begin{itemize}
%	\item new depth estimation CNNs achieve marginal gains, let's think about computational photography approaches that augment the measurements to achieve better performance
%	\item ...
%\end{itemize}
