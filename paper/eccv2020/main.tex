% updated April 2002 by Antje Endemann
% Based on CVPR 07 and LNCS, with modifications by DAF, AZ and elle, 2008 and AA, 2010, and CC, 2011; TT, 2014; AAS, 2016; AAS, 2020

\documentclass[runningheads]{llncs}
\usepackage{graphicx}
\usepackage{comment}
\usepackage{amsmath,amssymb} % define this before the line numbering.
\usepackage{color}

% INITIAL SUBMISSION - The following two lines are NOT commented
% CAMERA READY - Comment OUT the following two lines
\usepackage{ruler}
\usepackage[width=122mm,left=12mm,paperwidth=146mm,height=193mm,top=12mm,paperheight=217mm]{geometry}


\usepackage{booktabs}
\usepackage{caption}
\usepackage{array}
\usepackage{tabularx}
\usepackage{bm}
\usepackage{multirow}
\usepackage{float}
\usepackage[utf8x]{inputenc}
\usepackage{algorithm}
\usepackage{algpseudocode}
\usepackage{algorithmicx}
\usepackage{dblfloatfix}

\usepackage{mathtools}
\DeclarePairedDelimiter{\paren}{\lparen}{\rparen}
\DeclarePairedDelimiter{\bracket}{[}{]}
\DeclarePairedDelimiter{\ang}{\langle}{\rangle}
\DeclarePairedDelimiter{\abs}{\lvert}{\rvert}
\DeclarePairedDelimiter{\set}{\{}{\}}
\DeclarePairedDelimiter{\norm}{\|}{\|}
\DeclareMathOperator{\dom}{dom}
% The above let you do things like:
% Let $A = \set{1,2,\ldots,n}$...

\newtheorem{thm}{Theorem}
\newtheorem{lem}[thm]{Lemma}
\newtheorem{prop}[thm]{Proposition}
\newtheorem{cor}[thm]{Corollary}
\newtheorem{defn}{Definition}[section]
\newtheorem{conj}[thm]{Conjecture}

\newcommand{\alg}[1]{\begin{algorithm}\begin{algorithmic}[0]#1\end{algorithmic}\end{algorithm}}
\newcommand{\eqn}[1]{\begin{equation*}#1\end{equation*}}
\newcommand{\eqnsplit}[1]{\begin{equation}\begin{split}
#1
\end{split}\end{equation}}
\newcommand{\txt}[1]{\textrm{#1}}
\newcommand{\s}[1]{\section{#1}}
\newcommand{\sbs}[1]{\subsection{#1}}

\newcommand{\id}{\mathrm{id}}
\newcommand{\divs}{\mid}
\newcommand{\tdivs}{\,\Big\vert\,}
\newcommand{\ndivs}{\nmid}

\newcommand{\mset}[2]{\set*{\, #1 : #2 \,}}
\newcommand{\msset}[2]{\left\{\, #1 \;\middle\vert\; #2 \,\right\}}
\newcommand{\smat}[1]{\paren*{\begin{smallmatrix} #1 \end{smallmatrix}}}
\newcommand{\ord}[1]{\left| #1 \right|}
% Blackboard bold.
\newcommand{\N}{\mathbb{N}}
\newcommand{\Z}{\mathbb{Z}}
\newcommand{\Q}{\mathbb{Q}}
\newcommand{\R}{\mathbb{R}}
\newcommand{\F}{\mathbb{F}}
\newcommand{\E}{\mathbb{E}}

\newcommand{\X}{\mathcal{X}}

% Other
\newcommand{\del}{\partial}
\newcommand{\Real}{\text{Re}}
\newcommand{\Imag}{\text{Im}}
\newcommand{\res}{\text{res}}

% \DeclareRobustCommand\onedot{\futurelet\@let@token\@onedot}
% \def\@onedot{\ifx\@let@token.\else.\null\fi\xspace}

\def\eg{\emph{e.g.}} \def\Eg{\emph{E.g.}}
\def\ie{\emph{i.e.}} \def\Ie{\emph{I.e.}}
\def\cf{\emph{c.f.}} \def\Cf{\emph{C.f.}}
\def\etc{\emph{etc}} \def\vs{\emph{vs.}}
\def\wrt{w.r.t.} \def\dof{d.o.f.}
\def\etal{\emph{et al.}}

\newcommand{\edit}[1]{\textcolor{red}{#1}}

\graphicspath{{figures/}{"G:/Shared drives/Stanford Computational
    Imaging/Projects/single_spad_depth/figures/"}{"H:/Shared drives/Stanford Computational
    Imaging/Projects/single_spad_depth/figures/"}{"/Users/gordon/mount/teamdrive/Shared
    drives/Stanford Computational Imaging/Projects/single_spad_depth/figures/"}
  {"/Volumes/GoogleDrive/Shared drives/Stanford Computational Imaging/Projects/single_spad_depth/figures/"}}
% Pages are numbered in submission mode, and unnumbered in camera-ready

\begin{document}
% \renewcommand\thelinenumber{\color[rgb]{0.2,0.5,0.8}\normalfont\sffamily\scriptsize\arabic{linenumber}\color[rgb]{0,0,0}}
% \renewcommand\makeLineNumber {\hss\thelinenumber\ \hspace{6mm} \rlap{\hskip\textwidth\ \hspace{6.5mm}\thelinenumber}}
% \linenumbers
\pagestyle{headings}
\mainmatter
\def\ECCVSubNumber{3668}  % Insert your submission number here

%\title{Improving Monocular Depth Estimation with Global Depth Histogram Matching using a Single SPAD Transient} % Replace with your title
\title{\textcolor{red}{Disambiguating Monocular Depth Estimation with a Single Transient}} % Replace with your title

% INITIAL SUBMISSION 
%\begin{comment}
\titlerunning{ECCV-20 submission ID \ECCVSubNumber} 
\authorrunning{ECCV-20 submission ID \ECCVSubNumber} 
\author{Anonymous ECCV submission}
\institute{Paper ID \ECCVSubNumber}
%\end{comment}
%******************

% CAMERA READY SUBMISSION
\begin{comment}
\titlerunning{Abbreviated paper title}
% If the paper title is too long for the running head, you can set
% an abbreviated paper title here
%
\author{First Author\inst{1}\orcidID{0000-1111-2222-3333} \and
Second Author\inst{2,3}\orcidID{1111-2222-3333-4444} \and
Third Author\inst{3}\orcidID{2222--3333-4444-5555}}
%
\authorrunning{F. Author et al.}
% First names are abbreviated in the running head.
% If there are more than two authors, 'et al.' is used.
%
\institute{Princeton University, Princeton NJ 08544, USA \and
Springer Heidelberg, Tiergartenstr. 17, 69121 Heidelberg, Germany
\email{lncs@springer.com}\\
\url{http://www.springer.com/gp/computer-science/lncs} \and
ABC Institute, Rupert-Karls-University Heidelberg, Heidelberg, Germany\\
\email{\{abc,lncs\}@uni-heidelberg.de}}
\end{comment}
%******************
\maketitle

\begin{abstract}
  Monocular depth estimation algorithms successfully predict the
  relative depth {order} of objects in a scene.
  However, because of the fundamental scale ambiguity associated with monocular
  images, these algorithms fail at correctly predicting true metric
  depth. In this work, we demonstrate how a depth histogram of the scene, which can be readily captured using a \textcolor{red}{single-pixel time-resolved detector}, can be fused with the output of existing monocular depth estimation algorithms to resolve the depth ambiguity problem.
  We validate this novel sensor fusion technique experimentally and in extensive
  simulation. We show that it significantly improves the performance of several
  state-of-the-art monocular depth estimation algorithms.  
  \keywords{depth estimation, time-of-flight imaging}
\end{abstract}


\section{Introduction}
\label{sec:intro}
%
\subsection{Context}
Estimating depth from images is an important imaging problem, as dense depth
maps are useful precursors for high-level scene understanding tasks like
segmentation and pose estimation (cite cite cite) and mid-level perception (of
which depth estimation is an important part) has been shown to be very useful
for e.g. training robots to navigate their environments. While traditional approaches to depth
estimation use multiple cameras or structure-from-motion, convolutional neural
networks have also demonstrated reasonable performance on the so-called
monocular depth estimation task, where the network is trained to produce a dense
depth map given only a single RGB image of the scene.
\subsection{Problem Statement}
While deep monocular depth estimators have demonstrated strong performance (cite
cite) and even some generalizability across scene types, the task they are solving is fundamentally
underconstrained due to \textit{inherent scale ambiguity}, i.e. the unresolvable
tradeoff between size and distance in monocular images. This ambiguity could be
resolved by adding an additional camera and calculating a disparity map, but
this method still fails on textureless regions and areas with lots of
occlusions. Other approaches use FMCW or time-of-flight LiDAR technologies,
but these approaches are currently expensive and bulky. 
\subsection{Proposed approach}
In this paper, we show that by augmenting the RGB image with a histogram of
depth information, we can achieve substantially improved performance
(and generalizability) over state-of-the-art monocular depth
estimators. By performing an exact, weighted histogram matching on the output
depth map of the depth estimator, we can match the depth histogram of the scene
to the depth histogram of our estimate. Such a histogram can be captured
relatively inexpensively using only a single pixel single-photon avalanche diode
(SPAD) and pulsed laser illumination diffused over the field of view.
\subsection{Impact}
Our method is a lightweight postprocessing step that substantially improves
the quality of depth maps produced by monocular depth estimators. It can be
applied to any method to improve the accuracy instantly. (Our method also
helps neural-network-based methods generalize across scene types easily.)
\subsection{Limitations}
Our method is not without limitations, however. It still requires a laser and
single-pixel detector, and as such, is sensitive to ambient photons. Our method,
which is a fundamentally a variant of histogram matching, is, like histogram
matching, unable to transpose the values of pixels (i.e. if pixel $a$ is farther
than pixel $b$ in the input, it will be farther than pixel $b$ in the output).
In other words, our method is not able to resolve ordinal depth errors
(errors where an object is wrongly placed closer or farther
relative to another object). Finally, our method is non-differentiable, and is
therefore unsuitable for end-to-end optimization of multi-part networks.

\begin{itemize}
	\item We introduce the idea of augmenting an RGB camera with a single-pixel
    SPAD to address scale ambiguity error in monocular depth estimators.	
  \item We analyze our approach on indoor scenes using the NYU Depth v2 dataset.
    We demonstrate that our approach is able to resolve scale ambiguity while
    being fast and easy to implement.
  \item (Potentially) We investigate the ability of our method to 
    help generalization of monocular depth estimators across scene types. 
    We (hopefully) demonstrate that our method allows monocular depth
    estimators to perform well even on completely different scene types.
	\item We build a hardware prototype and evaluate the efficacy of our
    approach on real-world data. 
\end{itemize}




\section{Related Work}
\label{sec:related}
% \paragraph{Depth Imaging}
% Include in intro, don't necessarily need it here.
% Conventional approaches to estimating depth from images include stereo-based
% approaches 
% \begin{itemize}
% 	\item stereo and multiview
% 	\item structured illumination and random patterns (kinect, etc.), active stereo
% 	\item time of flight (continuous wave and pulsed)
% 	\item what we do: like pulsed but much simpler setup; no scanning, no spad array, ...
% \end{itemize}


%%%%%%%%%%%%%%%%%%%%%%%%%%%%%%%%%%%%%%%%%%%%%%%%%%%%%%%%%%%%%%%%%%%%%%%%%%%%%%%%%%%%%%%%%%%%%%%%%
\paragraph{Monocular Depth Estimation}
%
Estimating a depth map from a single RGB image has been approached using Markov Random Fields~\cite{Saxena2006}, geometric approaches~\cite{Hoiem2005}, and non-parametric, SIFT-based methods~\cite{Karsch2014}. More recently, deep neural networks have been applied to this problem. For example, Eigen et al.~\cite{Eigen2014} use a multi-scale neural network to
predict depth maps, Godard et al.~\cite{Godard2017} use an unsupervised approach that trains a network using stereo pairs, and Fu et al.~\cite{Fu2018} combine a logarithmic depth discretization scheme with an ordinal regression loss function. Various experiments using different types of encoder networks (\eg ResNet, DenseNet)~\cite{Alhashim2018,Laina2016} have also been employed with some success, as have approaches mixing deep learning with conditional random fields~\cite{Xu2017}, and attention-based approaches~\cite{Hao2018,Xu2018}. Recently, Lasinger et al.~\cite{Lasinger:2019} improved the robustness of monocular depth estimation using cross-dataset transfer.

Despite achieving remarkable success on estimating ordinal depth from a single image, none of these methods are able to resolve inherent scale ambiguity in a principled manner. We introduce a new approach that leverages existing monocular depth estimation networks and disambiguates the output using depth histogram-like measurements obtained from a single, diffused SPAD. Other approaches to disambiguating monocular depth estimation use optimized freeform lenses~\cite{Chang:2019:DeepOptics3D,Wu:2019} or dual-pixel sensors~\cite{Garg:2019}, but these approaches require custom lenses or sensors and specialized image reconstruction methods. In contrast, our approach is potentially compatible with existing camera systems (\eg deployed on current cell phones) and our algorithms could be used in tandem with existing camera ISPs.

%%%%%%%%%%%%%%%%%%%%%%%%%%%%%%%%%%%%%%%%%%%%%%%%%%%%%%%%%%%%%%%%%%%%%%%%%%%%%%%%%%%%%%%%%%%%%%%%%
\paragraph{Depth Imaging and Sensor Fusion with SPADs}
%
Emerging single-photon LiDAR systems use single-photon avalanche diodes (SPADs) to record the time of flight of individual photons. SPAD detectors can be fabricated using standard CMOS processes, but the required picosecond-accurate time-stamping electronics are challenging to miniaturize and fabricate at low cost. For this reason, many single-photon 3D imaging approaches use a single SPAD combined with a raster scanning mechanism~\cite{Kirmani:2014,Lamb2010,Li:2019,pawlikowska2017single}. Unfortunately, this makes it challenging to scan dynamic scenes at high resolution and scanners can also be expensive, difficult to calibrate, and prone to mechanical failure. To reduce the scanning complexity to one dimension, 1D SPAD arrays have been developed~\cite{burri2017linospad,burri2016linospad,OToole2017}, and 2D SPAD arrays are also an active area of research~\cite{Niclass2005,Stoppa2007,Veerappan2011,Zhang2018}. Yet, single-pixel SPADs remain the only viable option for low-cost consumer devices today.

The proposed method uses a single-pixel SPAD and pulsed light source that are diffused across the entire scene instead of aimed at a single point, as with proximity sensors. This unique configuration captures a measurement that closely resembles the depth histogram of the scene. Our sensor fusion algorithm achieves reliable absolute depth estimation by combining the SPAD measurement with the output of a monocular depth estimator using a histogram matching technique. While other recent work also explored RGB-SPAD sensor fusion~\cite{Lindell2018}, the RGB image was primarily used to guide the denoising and upsampling of measurements from a SPAD array.
 
%Previous work (see \cite{Horaud2016} for a survey) has been able to use single-pixel SPADs \cite{Lamb2010} and also 1D LinoSPADs in tandem with various scanning or DMD devices to capture 3D volumes of photon arrivals that can be used to reconstruct depth. Lindell et. al. \cite{Lindell2018} use a LinoSPAD and epipolar scanline and fuse the SPAD data with an RGB image to produce high-quality depth.  Our approach uses a single pixel SPAD but does not require any scanning or DMD mechanism.

%A parallel approach called 3D flash LiDAR uses a laser with an optical diffuser
%as the illumination source and a 2D array of SPADs to capture the 3D volume
%\cite{Stoppa2007, Niclass2005}. Such arrays are capable of reconstructing high
%quality depth but remain relatively low resolution. Other arrays are able to
%achieve higher resolution, but suffer from low fill factor \cite{Veerappan2011} or
%sacrifice per-pixel TDC \cite{Zhang2018}.

% \begin{itemize}
%   \item Scanned single-pixel and 1D arrays, but scanning is hard
%   \item high resolution arrays -> challenging to do TDC
%   \item Individual (non-scanning) SPADs already exist in e.g. iPhoneX, but use
%     is limited to proximity sensors.
%   \item What we do: No array (easier), no scanning (easier), much cheaper,
%     combined with RGB camera to do high-resolution depth imaging, a more complex
%     task.
% \end{itemize}

%%%%%%%%%%%%%%%%%%%%%%%%%%%%%%%%%%%%%%%%%%%%%%%%%%%%%%%%%%%%%%%%%%%%%%%%%%%%%%%%%%%%%%%%%%%%%%%%%
% \paragraph{Deep Sensor Fusion}
% Maybe mention in previous section? 
% global hints for super-resolution, colorization, depth estimation 
%
% \begin{itemize}
	% \item colorization
	% \item david's 2018 paper for depth estimation and denoising (see david's 2019 sig paper for related work)
	% \item what we do: slightly different application
% \end{itemize}

%%%%%%%%%%%%%%%%%%%%%%%%%%%%%%%%%%%%%%%%%%%%%%%%%%%%%%%%%%%%%%%%%%%%%%%%%%%%%%%%%%%%%%%%%%%%%%%%%
\paragraph{Histogram Matching and Global Hints}
%
Histogram matching is a well-known image processing technique for adjusting an image so that its histogram matches some pre-specified histogram (often derived from another image)~\cite{gonzales1977gray,Gonzalez2008}. Nikolova et al.~\cite{Nikolova2013} use optimization to recover a strict ordering of the image pixels, yielding an exact histogram match. Morovic et al.~\cite{Morovic2002} provide an efficient and precise method for fast histogram matching which supports weighted pixel values. In the image reconstruction space, Swoboda and Schn\"orr~\cite{Swoboda2013} use a histogram to form an image prior based on the Wasserstein distance for image denoising and inpainting. Rother et al.~\cite{Rother2006} use a histogram prior to create an energy function that penalizes foreground segmentations with dissimilar histograms. In a slightly different application area, Zhang et al.~\cite{Zhang2017} train a neural network to produce realistically colorized images given only a black-and-white image and a histogram of global color information.

In our procedure, the diffused SPAD measurements closely resemble a histogram of the depth map where the histogram values are weighted by spatially varying scene reflectances and inverse-square falloff effects. We therefore adapt the algorithm in Morovic et al.~\cite{Morovic2002} in order to accommodate general per-pixel weights during histogram matching.

% \textcolor{red}{Our method is essentially a modified form of the algorithm in \cite{Morovic2002}, modified for our particular use case. Also worth noting is the fact that most algorithms compute histograms from existing images, whereas our method mesaures the depth histogram indirectly using photon arrivals.} Note: this paragraph needs more work. We can say something like ``Inspired by Morovic et al., we do something'' but then we also need to highlight how our method is different. Perhaps concisely summarize how you adapt it for our SPAD model.
%
% \begin{itemize}
%   \item Exact histogram matching paper used in this work
%   \item Wasserstein-based optimization techniques for
%     histogram-based regularization
% \end{itemize}




\section{Method}
\label{sec:method}
\subsection{Monocular Depth Estimation} 

\subsection{Global Depth Hints} 

\begin{itemize}
	\item image formation model	
	\item training details
	\item simulations and comparison: qualitative and quantitative
\end{itemize}



\section{Evaluation and Assessment}
\label{sec:evaluation}



%%%%%%%%%%%%%%%%%%%%%%%%%%%%%%%%%%%%%%%%%%%%%%%%%%%%%%%%%%%%%%%%%%%%%%%%%%%%%%%%%%%%%%%%%%%%%%%
\subsection{Implementation Details}

We use the NYU Depth v2 dataset to evaluate our method. This dataset consists of
249~training and 215~testing RGB-D images captured with a Kinect.

To simulate a \edit{transient}, we take the provided depth map and 
calculate a weighted depth histogram by weighting the pixel contributions to
each depth bin by the luminance of each pixel. To model radiometric falloff, we
multiply each bin by $1/z^2$, and convolve with a modeled system temporal
response, which we approximate as a Gaussian with a full-width at half-maximum of 70~ps. We scale
the histogram by the total number of observed signal photon counts (set to 
$10^6$) and  add a fixed number of background photons $b \in \set{2\times 10^5, 10^5, 2\times
10^4, 10^4}$. The background counts are evenly distributed across all bins to simulate the ambient and dark
count detections, and the different background levels correspond to
signal-to-background ratios (SBR) of $5, 10, 50$ and $100$ respectively. Finally,
each bin is Poisson sampled to produce the final simulated transient.

%%%%%%%%%%%%%%%%%%%%%%%%%%%%%%%%%%%%%%%%%%%%%%%%%%%%%%%%%%%%%%%%%%%%%%%%%%%%%%%%%%%%%%%%%%%%%%%
\subsection{Simulated Results}
\begin{table*}[!t]
  \begin{center}
   {\fontsize{7pt}{8pt}\selectfont 
      \begin{tabular}{llllllll}
\toprule
           &                     & delta1 & delta2 & delta3 & rel\_abs\_diff &  rmse & log10 \\
model & hyperparams &        &        &        &              &       &       \\
\midrule
\multirow{7}{*}{dorn} & CNN &  0.846 &  0.954 &  0.983 &        0.120 & 0.501 & 0.053 \\
           & CNN + median rescaling &  0.871 &  0.964 &  0.988 &        0.111 & 0.473 & 0.048 \\
           & CNN + GT histogram matching &  0.906 &  0.972 &  0.990 &        0.095 & 0.419 & 0.040 \\
           & CNN + WHM intensity only (SBR=infinite) &  0.904 &  0.970 &  0.989 &        0.091 & 0.414 & 0.040 \\
           & CNN + WHM (SBR=10) &  0.903 &  0.970 &  0.989 &        0.091 & 0.422 & 0.040 \\
           & CNN + WHM (SBR=50) &  0.906 &  0.971 &  0.990 &        0.089 & 0.410 & 0.039 \\
           & CNN + WHM (SBR=100) &  0.906 &  0.971 &  0.990 &        0.090 & 0.408 & 0.039 \\
\cline{1-8}
\multirow{7}{*}{densedepth} & CNN &  0.847 &  0.973 &  0.994 &        0.123 & 0.461 & 0.053 \\
           & CNN + median rescaling &  0.888 &  0.978 &  0.995 &        0.106 & 0.409 & 0.045 \\
           & CNN + GT histogram matching &  0.930 &  0.984 &  0.995 &        0.079 & 0.338 & 0.034 \\
           & CNN + WHM intensity only (SBR=infinite) &  0.926 &  0.983 &  0.995 &        0.081 & 0.346 & 0.035 \\
           & CNN + WHM (SBR=10) &  0.922 &  0.982 &  0.994 &        0.082 & 0.361 & 0.036 \\
           & CNN + WHM (SBR=50) &  0.925 &  0.983 &  0.995 &        0.081 & 0.348 & 0.035 \\
           & CNN + WHM (SBR=100) &  0.926 &  0.983 &  0.995 &        0.081 & 0.346 & 0.035 \\
\bottomrule
\end{tabular}

   } 
\caption{
  Quantitative evaluation using NYU Depth v2. Bold indicates best
performance for that metric, while underline indicates second best. The proposed
scheme outperforms DenseDepth and DORN on all metrics, and it closely matches or
even outperforms the median rescaling scheme and histogram matching with the
exact depth map histogram, even though those methods have access to ground
truth.}
\label{tab:comparison}
\end{center}
\end{table*}

We show an extensive quantitative evaluation in Table~\ref{tab:comparison}.
Here, we evaluate three recent monocular depth estimation CNNs:
DORN~\cite{Fu2018}, DenseDepth~\cite{Alhashim2018}, and
MiDaS~\cite{Lasinger:2019}. To evaluate the quality of DORN and DenseDepth, we
report various standard error metrics~\cite{Eigen2014}. Moreover, we show a simple
post-processing step that rescales their outputs to match the median ground truth
depth~\cite{Alhashim2018}. We also show the results of histogram matching the
output of the CNNs with the ground truth depth map histogram. Note that we do
not report the quality of the direct output of MiDaS as this algorithm does not
output metric depth. However, we do show its output histogram matched with the
ground truth depth map histogram. In all cases, post-processing the estimated
depth maps either with the median depth or depth histogram significantly
improves the absolute depth estimation, often by a large margin compared to the
raw output of the CNNs. Unfortunately, ground truth depth is typically not
accessible so neither of these two post-processing methods are viable in
practical application scenarios.

Instead, our method uses the simulated measurements from a single \edit{aggregated transient} to correct the depth map. In Table~\ref{tab:comparison}, results are shown for several different
signal-to-background ratios (SBRs). We see that the proposed method achieves
high-quality results for correcting the raw depth map estimated by the
respective CNNs for all cases. The quality of the resulting depth maps is almost
as good as that achieved with the oracle ground truth histogram, which can be
interpreted as an approximate upper bound on the performance, despite a
relatively high amount of noise and background signal. These results demonstrate
that the proposed method is agnostic to the specific depth estimation CNN
applied to get the initial depth map and that it generally achieves significant
improvements in the estimated depth maps, clearly surpassing the variation in
performance between depth estimation CNNs.

\begin{figure*}[t]
  % \includegraphics[width=\textwidth]{comparison/densedepth_468_comparison.pdf}
  % \includegraphics[width=\textwidth]{comparison/densedepth_194_comparison.pdf}
  \includegraphics[width=\textwidth]{comparison.pdf}
  %\includegraphics[width=\textwidth]{comparison/densedepth_258_comparison.png}
  \caption{Simulated results from NYU Depth v2 computed with the DenseDepth
  CNN~\cite{Alhashim2018}. The depth maps estimated by the CNN are reasonable,
  but contain systematic error. Oracle access to the ground truth depth maps,
  either through the median depth or the depth histogram, can remove this error
  and correct the depth maps. \edit{The proposed method uses a single transient measurement} and does not rely on ground truth depth, but it achieves
  a quality that closely matches the best-performing oracle.}
	\label{fig:results_simulated}
  \vspace{-1em}
\end{figure*}


In Figure~\ref{fig:results_simulated}, we also show qualitative results of our
simulations. For each of these scenes, we show the RGB reference image, the
ground truth depth map, the raw output of the DenseDepth CNN, the result of
rescaling the CNN output with the median ground truth depth, the result of
histogram-matching the CNN output by the ground truth depth map histogram, and
the result achieved by the proposed method for an SBR of 100. Error maps for all
the depth estimation methods are shown. As expected, the CNN outputs depth maps
that look reasonable but that have an average root mean squared error (RMSE) of
about 50--60~cm. Rescaling this depth map to match the median ground truth depth
value slightly improves the quality and histogram-matching with the ground
truth depth histogram shows a large amount of improvement. The quality of the
proposed method is close to using the oracle histogram, despite relying on
noisy \edit{transient} measurements. Additional simulations using DenseDepth and other depth
estimation CNNs for a variety of scenes are shown in the supplement.




\section{Experimental Demonstration}
\label{sec:prototype}
\begin{figure}[H]
  \includegraphics[width=\linewidth]{prototype_single_col.pdf}
  \caption{Prototype scanning setup. The pulsed light from the laser travels
    through a beam splitter before being guided by the galvo to the scene.
    Returning light is measured by the single-pixel SPAD. The RGB camera of a
    Kinect v2 is used to capture the monocular RGB image (the depth camera is
    not used)}
  \label{fig:prototype}
\end{figure}

%%%%%%%%%%%%%%%%%%%%%%%%%%%%%%%%%%%%%%%%%%%%%%%%%%%%%%%%%%%%%%%%
\subsection{Prototype RGB-SPAD Camera Hardware}

% The laser operates at 450~nm with a pulse repetition rate of 25~MHz with a peak power of 450~mW and average power of 0.5~mW.

As shown in Figure~\ref{fig:prototype}, our prototype camera comprises a color camera (Microsoft Kinect v2), a single-pixel SPAD (Micro Photon Devices 100~$\mu m$ PDM series, free-running), a laser (ALPHALAS PICOPOWER-LD-450-50), and a two-axis galvanometer mirror system (Thorlabs GVS012). The laser operates at 670~nm with a pulse repetition rate of 10~MHz with a peak power of 450~mW and average power of 0.5~mW. The SPAD records temporal histograms with 65536 bins, each corresponding to a time window of 4~ps. SPAD and laser are co-axially aligned using a beam splitter (Thorlabs PBS251). The full width at half maximum (FWHM) of the combined laser pulse width and SPAD jitter is about 70~ps, allowing the system to record depth map with an accuracy of about 1~cm. A National Instruments data acquisition device (NI-DAQ USB-6343) provides synchronization signals for the galvos, SPAD, and laser. The ground truth depth map is raster-scanned at a resolution of $512 \times 512$ pixels. The single-pixel, diffused SPAD measurement is generated by summing all of these measurements for a specific scene. This allows us to validate the accuracy of the proposed histogram matching algorithm that only uses the single histogram of the diffused SPAD and compare it with ground truth. Monocular depth estimation is calculated using the RGB image captured by the Kinect v2.

We determined camera intrinsics and extrinsics for the Kinect's RGB camera and
the scanning system using the standard camera calibration toolbox in MATLAB.
In general it is impossible to compensate for the offset between camera and
SPAD, but we do add the z displacement from spad to kinect to partially
compensate for it.
% \textcolor{red}{Mark, please write a short paragraph on calibration details, including any warping of the SPAD histograms you did to compensate for the offset in camera and SPAD position.}

Missing:
%
\begin{itemize}
\item RGB resolution used for MDE
\item do we also have Kinect depth maps for comparison? (yes) kinect resolution: RGB is 1920x1080 and depth camera is 512x424
\end{itemize}

%%%%%%%%%%%%%%%%%%%%%%%%%%%%%%%%%%%%%%%%%%%%%%%%%%%%%%%%%%%%%%%%
\subsection{Experimental Results}


\section{Discussion}
\label{sec:discussion}
In summary, we demonstrate a method to greatly improve depth estimates from
monocular depth estimators by correcting the scale ambiguity errors inherent
with such techniques. Our approach produces depth maps with accurate absolute
depth, and helps MDE neural networks generalize across scene
types, including on data captured with our hardware prototype.  Moreover, we
require only minimal additional sensing hardware; we show that a single
measurement histogram from a diffused SPAD sensor contains enough information
about global scene geometry to correct errors in monocular depth estimates.

The performance of our method is highly dependent on the accuracy of the initial
depth map of the MDE algorithm. Our results demonstrate that when the MDE
technique produces a depth map with good ordinal accuracy, where the ordering of
object depths is roughly correct, the depth estimate can be corrected to produce
accurate absolute depth. However, if the ordering of the initial depths is not
correct, these errors may propagate to the final output depth map.

In the optically diffused configuration, the laser
power is spread out over the entire scene. Accordingly, for distant scene points
very little light may return to the SPAD, \edit{ making reconstruction difficult
(an analogous problem occurs with dark objects).} \edit{Thus, our method is
best suited to short- to medium-range scenes.}
\edit{On the other hand, in bright environments, pileup will ultimately limit the range of our
  method. However, this can be mitigated with optical elements to reduce the amount
  of incident light, with pileup correction~\cite{Heide:2018,Rapp:2019}, or
  even by taking two transient measurements, one with and one without laser
  illumination, and using their difference to approximate the background-free
  transient.}
\edit{Finally, under normal indoor conditions, it is theoretically possible to
  achieve an SBR of 5 at a range of 3 meters with a laser of only 21 mW while
  remaining in the low-flux regime. We confirm this empirically with our
  diffused setup, which operates without significant pileup effects while using
  approximately 25 mW of laser power (see
  Figure~\ref{fig:captured_diffuse_small} and the supplement for details).}



  %Note
  % that this calculation involves only on the signal-to-background photon ratio,
  % which is independent of the actual number of counts recorded per exposure. In
  % other words, pileup and laser energy concerns may 
  % be addressed entirely independently. As for acquisition time, at a 10 MHz
  % pulse rate with
  % a 5\% photon detection rate, it takes just two seconds to acquire a million
  % photon events.} 

% Furthermore, our system uses a SPAD and is therefore susceptible to pileup.
% Because the SPAD must be quenched before it can
% detect another photon, earlier photons
% from closer objects may systematically prevent detection of photons from further
% objects, causing strong distortion of the transient. Fortunately, at the expense of
% longer acquisition times, pileup can be mitigated
% to a large extent by operating in the low-flux regime
% where the number of photon events per second is controlled to be at most 5\% the
% number of laser pulses per second~\cite{oconnortcspc}. Finally, even in
% the high-flux regime, it is sometimes possible to correct pileup computationally~\cite{Heide:2018}.}

% While our prototype shows results for scanned/summed measurements, our scanning prototype operates in the low-flux regime,
% avoiding pileup. A hypothetical diffused SPAD prototype could also operate in
% this low-flux regime, which would yield identical measurement models and identical
% results. Figure~\ref{fig:hardware}(b)-(c) empirically shows that collecting and summing scanned
% measurements is nearly identical to capturing diffused measurements. ?!We show
% empirical comparisons between the two modes on a more complex scene in the supplement?!}

% Even if diffused and scanned/summed measurements are equivalent
%   computationally, they are not equivalent energetically.  This number is dependent on scene albedo, but all time-of-flight imaging
% systems share this dependency.}


%%%%%%%%%%%%%%%%%%%%%%%%%%%%%%%%%%%%%%%%%%%%%%%%%%%%%%%%%%%%%%%%%%%%%%%%%%%%%%%%%%%%
\paragraph{Future Work}
\edit{Future work could implement our algorithm or similar sensor fusion
  algorithms on smaller platforms such as existing cell phones with single-pixel SPAD
proximity sensors and RGB cameras. Necessary
  adjustments, such as pairing near-infrared (NIR) SPADs with NIR sensors,
  could be made. The small baseline of such sensors
  would also mitigate the effects of shading and complex BRDFs on the
  reflectance
  estimation step. More sophisticated intrinsic imaging techniques could also be
  employed.}

% \edit{DELETE Learning-based methods to combine the MDE estimates and
% histogram. One might even consider sensing regimes where the number of
% returning signal photons is low, such as when the time-resolved detector and
% camera operate at high framerates. While most MDE techniques are tailored to
% clean RGB images, the transient could be used to help MDE techniques generalize
% to noisy scenes under low-light conditions.}


%%%%%%%%%%%%%%%%%%%%%%%%%%%%%%%%%%%%%%%%%%%%%%%%%%%%%%%%%%%%%%%%%%%%%%%%%%%%%%%%%%%%
\paragraph{Conclusions}
Since their introduction, monocular depth estimation algorithms have improved
tremendously.  However, recent advances, which have generally relied on new
network architectures or revised training procedures, have produced only modest
performance improvements. In this work we dramatically improve the performance
of several monocular depth estimation algorithms by fusing their estimates with
transient measurements. Such histograms are easy to capture using time-resolved
single-photon detectors and are poised to become an important component of
future low-cost imaging systems.

%Moving forward, sensor fusion is poised to become an even more important component of a variety of vision systems.
%Computational photography approaches may provide the measurements needed to achieve much greater performance improvements, especially as the sensors in devices we use every day become more diverse, complex, and increasingly rely on techniques based on sensor fusion and sophisticated computational algorithms.
 
%\begin{itemize}
%	\item new depth estimation CNNs achieve marginal gains, let's think about computational photography approaches that augment the measurements to achieve better performance
%	\item ...
%\end{itemize}


\clearpage
% ---- Bibliography ----
%
% BibTeX users should specify bibliography style 'splncs04'.
% References will then be sorted and formatted in the correct style.
%
\bibliographystyle{splncs04}
\bibliography{references}

\end{document}

