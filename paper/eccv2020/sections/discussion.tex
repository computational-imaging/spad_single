In summary, we demonstrate a method to greatly improve depth estimates from
monocular depth estimators by correcting the scale ambiguity errors inherent
with such techniques.  Our approach produces depth maps with accurate absolute
depth, and helps the generalization of neural networks for MDE across scene
types, including on data captured with our hardware prototype.  Moreover, we
require only minimal additional sensing hardware; we show that a single measurement
histogram from a diffused SPAD sensor contains enough information about global
scene geometry to correct errors in monocular depth estimates.

%%%%%%%%%%%%%%%%%%%%%%%%%%%%%%%%%%%%%%%%%%%%%%%%%%%%%%%%%%%%%%%%%%%%%%%%%%%%%%%%%%%%
\paragraph{Limitations}
The performance of our method is highly dependent on the accuracy of the initial
depth map of the MDE algorithm.
%Since our method relies on the monocular depth estimate, the accuracy of the output depth map also relies on the performance of the initial depth estimate.
Our results demonstrate that when the MDE technique produces a depth map
with good ordinal accuracy, where the ordering of object depths is roughly correct, the
depth estimate can be corrected to produce accurate absolute depth. However, if
the ordering of the initial depths is not correct, these errors will not be
corrected by our histogram matching procedure and may propagate to the final
output depth map.

Additionally, since our method relies on a diffused pulsed laser, the laser
power is spread out over the entire scene. Accordingly, for distant scene points very
little light will return to the SPAD and it may be difficult to accurately
capture distant scene geometry in the histogram. Thus our method is best suited
to short to medium-range scenes. \edit{For example, assume an indoor scene with
fluorescent bulbs and an ambient spectral irradiance of $I_A =
2~\text{mW}/\text{m}^2$ (across the 1 nm pass band of a spectral filter matched
to the laser). The laser power required to achieve a minimum SBR of 5 for a
diffuse scene at $r = 2$ m
and a field of view of $\theta = 40^\circ$ can be calculated as}
\begin{equation}
  P_{\text{min}} = I_A \cdot 4 r^2 \tan^2(\theta/2) \cdot SBR_{\text{min}},
\end{equation} 
\edit{giving $P_{\text{min}}=21$ mW, less than the 60 mW of the Kinect.}


%%%%%%%%%%%%%%%%%%%%%%%%%%%%%%%%%%%%%%%%%%%%%%%%%%%%%%%%%%%%%%%%%%%%%%%%%%%%%%%%%%%%
\paragraph{Future Work}
While our hardware prototype is large, future work could miniaturize this
system. Our algorithm or similar sensor fusion algorithms could
also be integrated into electronics that already contain the required hardware
components, for example, existing cell phones with single-pixel SPAD
proximity sensors and RGB cameras.

Other methods for extracting scene information from the SPAD histogram could be
employed, including learning-based methods to combine the MDE estimates and
histogram. One might even consider sensing regimes where the number of
returning signal photons is low, such as when the SPAD and camera operate
at high framerates. While most MDE techniques are tailored to clean RGB images,  
the SPAD histogram could be used to help MDE
techniques generalize to noisy scenes under low-light conditions. 
 


%%%%%%%%%%%%%%%%%%%%%%%%%%%%%%%%%%%%%%%%%%%%%%%%%%%%%%%%%%%%%%%%%%%%%%%%%%%%%%%%%%%%
\paragraph{Conclusions}
Since their introduction, monocular depth estimation algorithms have improved tremendously.  However, recent advances, which have generally relied on new network architectures or revised training procedures, have produced only modest performance improvements. 
In this work we dramatically improve the performance of several monocular depth estimation algorithms by fusing their estimates with depth histogram measurements. 
Such histograms are easy to capture using SPADs and are poised to become an important component of future low-cost imaging systems.
%Moving forward, sensor fusion is poised to become an even more important component of a variety of vision systems.
%Computational photography approaches may provide the measurements needed to achieve much greater performance improvements, especially as the sensors in devices we use every day become more diverse, complex, and increasingly rely on techniques based on sensor fusion and sophisticated computational algorithms.
 
%\begin{itemize}
%	\item new depth estimation CNNs achieve marginal gains, let's think about computational photography approaches that augment the measurements to achieve better performance
%	\item ...
%\end{itemize}
