\documentclass[10pt,twocolumn,letterpaper]{article}
\usepackage{cvpr}
\usepackage{times}
\usepackage{epsfig}
\usepackage{graphicx}
\usepackage{amsmath}
\usepackage{amssymb}

% Include other packages here, before hyperref.
\usepackage{float}
% If you comment hyperref and then uncomment it, you should delete
% egpaper.aux before re-running latex.  (Or just hit 'q' on the first latex
% run, let it finish, and you should be clear).
\usepackage[pagebackref=true,breaklinks=true,letterpaper=true,colorlinks,bookmarks=false]{hyperref}

%%%%%%%%% PAPER ID  - PLEASE UPDATE
\def\cvprPaperID{2224} % *** Enter the CVPR Paper ID here
\def\httilde{\mbox{\tt\raisebox{-.5ex}{\symbol{126}}}}
\raggedbottom
\begin{document}
\title{Optimizing MDE with Global Depth Histogram
  Matching using a Single SPAD Transient}

\maketitle
\thispagestyle{empty}
%%%%%%%%% BODY TEXT - ENTER YOUR RESPONSE BELOW
We thank our reviewers for their constructive feedback and will make every
effort to address their concerns.

We propose a technique for augmenting the capabilities of monocular depth
estimation using transient data from a single-pixel SPAD and diffuse laser
illumination using a straightforward histogram matching algorithm. We agree with
all of our reviewers that the idea of using such a depth transient is novel and
worth investigating.

\vspace{1mm}\noindent\textbf{Equivalence of scanning and diffuse SPAD (R1, R2)}
First, for
both scanned and diffuse modes, pileup can be corrected \cite{heide2018subpico} or avoided by operating in the
low-flux regime. 
Second, we empirically validate that the digitally summed scanned SPAD
measurements and the diffused measurements are almost identical. We will add
discussions of both of these points to the paper.
\vspace{-2mm}
\begin{figure}[H]
  \centering
  \includegraphics[width=0.8\linewidth]{figures/figure.pdf}
  \caption{Setup and captured transients for scanned and diffuse mode.}
\end{figure}

\vspace{-3mm}\noindent\textbf{Energy of diffuse SPAD (R1, R2, R3)}
Assuming an ambient spectral irradiance of $I_A = 0.1$ $mW/m^2$, we can calculate the
laser power required to achieve a minimum SBR of 5 for a diffuse scene at $r = 2$ m
and a field of view of $\theta = 40^\circ$ as
\begin{equation}
  P_{\text{min}} = I_A \cdot 4 r^2 \tan^2(\theta/2) \cdot SBR_{\text{min}}
\end{equation}
Which gives $P_{\text{min}} = 1.1$ W. laserpointersafety.com gives the Nominal
Ocular Hazard Distance for this laser as 0.34m, i.e. the laser light level is
safe beyond this distance. Given that there are VCELs with well above
this capability, we feel that our method is a very competitive, eye safe, option for short
range indoor scenes. 

% McManamon 2012 provide the following equation for any LiDAR setup:
% \begin{equation}
%   P_R = P_T \cdot \frac{\rho_tA_t}{A_{\text{illum}}} \cdot \frac{A_{\text{rec}}}{\pi R^2} \cdot \eta_{\text{atm}}^2 \cdot \eta_{\text{sys}}
% \end{equation}
%   In the diffused setting, With a laser at 450 nm, and under the assumptions of $P_T=0.0003, \rho_t = 0.5,
%   A_t = A_{\text{illum}}, A_{\text{rec}} = 7\times 10^{-8}$ (corresponding to a
%   SPAD with aperture diameter $300$ $\mu$m), $R = 5, \eta_{\text{atm}} = 1$, and
%   $\eta_{\text{sys}} = 0.5$ we
%   accumulate $10^6$ signal photons in 0.04 sec, 0.8 sec in the low-flux regime.
%   At 10MHz, this corresponds to 8 million pulses, or the equivalent of scanning
%   each point in a $512 \times 512$ grid 30 times. (Include ambient flux levels?)

\vspace{1mm}\noindent\textbf{Justification for two-step method (R1)} We believe
our final architecture's strength is the fact that it is agnostic to the
MDE strategy used and will note this in the paper.

\vspace{1mm}\noindent\textbf{Kinect/Dual Pixels (R2)} While each of these methods has its own
benefits and disadvantages, our approach adds minimal additional hardware to a
single RGB camera. We will rephrase our comparisons to these methods accordingly.

\vspace{1mm}\noindent\textbf{Failure/limited improvement cases (R2)} We show at least one example in
the supplement where our method improves in RMSE by only 0.062. In general, our
method struggles with scenes where the inital depth map contains ordinal depth
errors.

\vspace{1mm}\noindent\textbf{Maximum SBR used for simulated results (R2)} The
experimental results are intended to show the performance of the method under
more realistic conditions. For example, the SBR of the outdoor scene was
apprximately 0.15. Detailed statistics for the experimental scenes will be
included to help strengthen our argument for practicality.

\vspace{1mm}\noindent\textbf{Interreflections (R2)} We neglect interreflections
as a second-order effect following existing work on flash LiDAR systems. \cite{iddan20013dstudio}

\vspace{1mm}\noindent\textbf{Uniform background assumption and negative
  histogram values (R2)} When operating in the low-flux condition, we follow the
many papers \cite{lindell2018} which model background photons as a constant term that adds equally to all bins.
Also, our algorithm clips all negative histogram values to zero after the background
subtraction step and we will update the section to reflect this.  

\vspace{1mm}\noindent\textbf{Number of acquisition cycles (R1)} In simulation, the histogram
is directly computed via equation (2).
Scan times varied from 5 to 25 mins indoors and 3 mins for the outdoor
scene (the number of $512 \times 512$
frames can be computed from the scan time and the laser pulse rate of 10MHz).
While this seems excessively long, this was a deliberate decision to maximize the fidelity of our
ground truth reconstructions, not to simulate the real capture setting. We
will report the scan times and number of acquisitions in the paper.

\vspace{1mm}\noindent\textbf{Selection of $K$, $\ell$, and $u$ (R1)} For both simulation and
experiment, $K$ is chosen to tradeoff between speed and accuracy. For
$h_{\text{source}}$, $(\ell, u)$ are set to encompass all possible scene depths. For 
    $h_{\text{target}}$, $(\ell, u)$ are chosen based on the depth values for
    the bins $\hat n_{\text{first}}$ and $\hat n_{\text{last}}$, respectively.
    We will update the paper to clarify this point.

\vspace{1mm}\noindent\textbf{Use of overdramatic or imprecise language (R1, R2)} We will make various
wording adjustments, such as using ``moderate'' instead of
``slight'' to describe the improvement from median rescaling, using
``augmenting'' instead of ``optimizing'', and changing improper uses of
``histogram'' to ``transient''.

\vspace{1mm}\noindent\textbf{Teaser colormaps misleadingly scaled (R2)} We will adjust both
    figures to use the same scale.

\vspace{1mm}\noindent\textbf{Approximating Poisson dist. with Gaussian (R3)} We will add a
    footnote explaining our usage of this approximation where it occurs.

\vspace{1mm}\noindent\textbf{Typo in L388 (R3)} We will correct $n < n_{\text{last}}$ to $n <
    n_{\text{first}}$.

{\small
\bibliographystyle{ieeetr}
\bibliography{references}
}

\end{document}
