% \paragraph{Depth Imaging}
% Include in intro, don't necessarily need it here.
% Conventional approaches to estimating depth from images include stereo-based
% approaches 
% \begin{itemize}
% 	\item stereo and multiview
% 	\item structured illumination and random patterns (kinect, etc.), active stereo
% 	\item time of flight (continuous wave and pulsed)
% 	\item what we do: like pulsed but much simpler setup; no scanning, no spad array, ...
% \end{itemize}


%%%%%%%%%%%%%%%%%%%%%%%%%%%%%%%%%%%%%%%%%%%%%%%%%%%%%%%%%%%%%%%%%%%%%%%%%%%%%%%%%%%%%%%%%%%%%%%%%
\paragraph{Monocular Depth Estimation}
Previous non-deep approaches used Markov Random Fields \cite{Saxena2006},
geometric approaches \cite{Hoiem2005}, and non-parametric, SIFT-based methods
\cite{Karsch2014}.
More recently, deep neural networks have been applied to the problem of estimating depth from a
single image. Eigen et. al. \cite{Eigen2014} use a multi-scale neural network to
predict depth at multiple scales. Godard et. al. \cite{Godard2017} use an
unsupervised approach (i.e. that does not require ground truth depth) that
trains a network using stereo pairs to produce disparity maps from
single images, which can then be used to recover the depth. Fu et. al.
\cite{Fu2018} combined a logarithmic depth discretization scheme with a novel
ordinal regression approach. Various experiments using different types of
encoder networks (e.g. ResNet, DenseNet) \cite{Laina2016} \cite{Alhashim2018}
have also been employed with some success, as have approaches mixing deep
learning with conditional random fields \cite{Xu2017}, and attention-based
approaches \cite{Xu2018} \cite{Hao2018}.

Despite achieving remarkable success on the monocular depth estimation task, none of these methods are able to
resolve inherent scale ambiguity in a principled manner (being monocular in nature). By combining a monocular
depth estimator with a depth histogram, our method is able to do so.

% \begin{itemize}
% 	\item summary of architectures and cost functions: u-net type architecture with reverse huber loss
% 	\item what we do: same thing, but augment with global hints (inspired by these approaches, we do ...)
% \end{itemize}

%%%%%%%%%%%%%%%%%%%%%%%%%%%%%%%%%%%%%%%%%%%%%%%%%%%%%%%%%%%%%%%%%%%%%%%%%%%%%%%%%%%%%%%%%%%%%%%%%
\paragraph{Depth Imaging and Sensor Fusion with SPADs}
Previous work (see \cite{Horaud2016} for a survey) has been able to use
single-pixel SPADs \cite{Lamb2010} and also 1D LinoSPADs in
tandem with various scanning or DMD devices to capture 3D volumes of photon arrivals
that can be used to reconstruct depth. Lindell et. al. \cite{Lindell2018} use a
LinoSPAD and epipolar scanline and fuse the SPAD data with an RGB image to
produce high-quality depth.  Our approach uses a single pixel SPAD but
does not require any scanning or DMD mechanism.

A parallel approach called 3D flash LiDAR uses a laser with an optical diffuser
as the illumination source and a 2D array of SPADs to capture the 3D volume
\cite{Stoppa2007, Niclass2005}. Such arrays are capable of reconstructing high
quality depth but remain relatively low resolution. Other arrays are able to
achieve higher resolution, but suffer from low fill factor \cite{Veerappan2011} or
sacrifice per-pixel TDC \cite{Zhang2018}.

Our approach uses flash LiDAR but requires only a single pixel 
sensor.
(Still need to flesh out this section more with some higher resolution SPAD arrays)
% \begin{itemize}
%   \item Scanned single-pixel and 1D arrays, but scanning is hard
%   \item high resolution arrays -> challenging to do TDC
%   \item Individual (non-scanning) SPADs already exist in e.g. iPhoneX, but use
%     is limited to proximity sensors.
%   \item What we do: No array (easier), no scanning (easier), much cheaper,
%     combined with RGB camera to do high-resolution depth imaging, a more complex
%     task.
% \end{itemize}

%%%%%%%%%%%%%%%%%%%%%%%%%%%%%%%%%%%%%%%%%%%%%%%%%%%%%%%%%%%%%%%%%%%%%%%%%%%%%%%%%%%%%%%%%%%%%%%%%
% \paragraph{Deep Sensor Fusion}
% Maybe mention in previous section? 
% global hints for super-resolution, colorization, depth estimation 
%
% \begin{itemize}
	% \item colorization
	% \item david's 2018 paper for depth estimation and denoising (see david's 2019 sig paper for related work)
	% \item what we do: slightly different application
% \end{itemize}

%%%%%%%%%%%%%%%%%%%%%%%%%%%%%%%%%%%%%%%%%%%%%%%%%%%%%%%%%%%%%%%%%%%%%%%%%%%%%%%%%%%%%%%%%%%%%%%%%
\paragraph{Histogram Matching and Global Hints}
Histogram matching or histogram specification is a well-known image processing
technique \cite{Gonzalez2008} for adjusting an image so that it's histogram
matches some pre-specified histogram (often derived from another image).
Nikolova et. al. \cite{Nikolova2013} use optimization to recover a strict
ordering of the image pixels that allows an exact histogram match to be
obtained. Morovic et. al \cite{Morovic2002} provide a simple and concise method
that also achieves an exact histogram match while being very fast.
In the image reconstruction space, Swoboda and Schnörr \cite{Swoboda2013} use a
histogram to form an image prior based on the Wasserstein distance for image
denoising and inpainting. Rother et. al. \cite{Rother2006} use a histogram prior
to create an energy function that penalizes foreground segmentations with
dissimilar histograms. In a slightly different vein, Zhang et. al.
\cite{Zhang2017} train a neural network to produce realistically colorized
images given only a black-and-white image and a histogram of global color information.

Our method is essentially a modified form of the algorithm in
\cite{Morovic2002}, modified for our particular use case. Also worth noting is
the fact that most algorithms compute histograms from existing images, whereas
our method mesaures the depth histogram indirectly using photon arrivals.
%
% \begin{itemize}
%   \item Exact histogram matching paper used in this work
%   \item Wasserstein-based optimization techniques for
%     histogram-based regularization
% \end{itemize}


