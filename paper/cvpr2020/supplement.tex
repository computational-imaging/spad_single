\documentclass[10pt,letterpaper]{article}

\usepackage{cvpr}
\usepackage{times}
\usepackage{epsfig}
\usepackage{graphicx}
\usepackage{amsmath}
\usepackage{amssymb}

% Include other packages here, before hyperref.
\usepackage{booktabs}
\usepackage{caption}
\usepackage{subcaption}
\usepackage{array}
\usepackage{tabularx}
\usepackage{bm}
\usepackage{multirow}
\usepackage{float}
\usepackage[utf8x]{inputenc}
\usepackage{algorithm}
\usepackage{algpseudocode}
\usepackage{algorithmicx}

\usepackage{dblfloatfix}

\usepackage{mathtools}
\DeclarePairedDelimiter{\paren}{\lparen}{\rparen}
\DeclarePairedDelimiter{\bracket}{[}{]}
\DeclarePairedDelimiter{\ang}{\langle}{\rangle}
\DeclarePairedDelimiter{\abs}{\lvert}{\rvert}
\DeclarePairedDelimiter{\set}{\{}{\}}
\DeclarePairedDelimiter{\norm}{\|}{\|}
\DeclareMathOperator{\dom}{dom}
% The above let you do things like:
% Let $A = \set{1,2,\ldots,n}$...

\newtheorem{thm}{Theorem}
\newtheorem{lem}[thm]{Lemma}
\newtheorem{prop}[thm]{Proposition}
\newtheorem{cor}[thm]{Corollary}
\newtheorem{defn}{Definition}[section]
\newtheorem{conj}[thm]{Conjecture}

\newcommand{\alg}[1]{\begin{algorithm}\begin{algorithmic}[0]#1\end{algorithmic}\end{algorithm}}
\newcommand{\eqn}[1]{\begin{equation*}#1\end{equation*}}
\newcommand{\eqnsplit}[1]{\begin{equation}\begin{split}
#1
\end{split}\end{equation}}
\newcommand{\txt}[1]{\textrm{#1}}
\newcommand{\s}[1]{\section{#1}}
\newcommand{\sbs}[1]{\subsection{#1}}

\newcommand{\id}{\mathrm{id}}
\newcommand{\divs}{\mid}
\newcommand{\tdivs}{\,\Big\vert\,}
\newcommand{\ndivs}{\nmid}

\newcommand{\mset}[2]{\set*{\, #1 : #2 \,}}
\newcommand{\msset}[2]{\left\{\, #1 \;\middle\vert\; #2 \,\right\}}
\newcommand{\smat}[1]{\paren*{\begin{smallmatrix} #1 \end{smallmatrix}}}
\newcommand{\ord}[1]{\left| #1 \right|}
% Blackboard bold.
\newcommand{\N}{\mathbb{N}}
\newcommand{\Z}{\mathbb{Z}}
\newcommand{\Q}{\mathbb{Q}}
\newcommand{\R}{\mathbb{R}}
\newcommand{\F}{\mathbb{F}}
\newcommand{\E}{\mathbb{E}}

\newcommand{\X}{\mathcal{X}}

% Other
\newcommand{\del}{\partial}
\newcommand{\Real}{\text{Re}}
\newcommand{\Imag}{\text{Im}}
\newcommand{\res}{\text{res}}


% If you comment hyperref and then uncomment it, you should delete
% egpaper.aux before re-running latex.  (Or just hit 'q' on the first latex
% run, let it finish, and you should be clear).
\usepackage[breaklinks=true,bookmarks=false]{hyperref}

%\cvprfinalcopy % *** Uncomment this line for the final submission

\graphicspath{{"G:/Shared drives/Stanford Computational
    Imaging/Projects/single_spad_depth/figures/"}{"H:/Shared drives/Stanford Computational
    Imaging/Projects/single_spad_depth/figures/"}{"/Users/gordon/mount/teamdrive/Shared
    drives/Stanford Computational Imaging/Projects/single_spad_depth/figures/"}
  {"/Volumes/GoogleDrive/Shared drives/Stanford Computational Imaging/Projects/single_spad_depth/figures/"}}

\def\cvprPaperID{****} % *** Enter the CVPR Paper ID here
\def\httilde{\mbox{\tt\raisebox{-.5ex}{\symbol{126}}}}

% Pages are numbered in submission mode, and unnumbered in camera-ready
%\ifcvprfinal\pagestyle{empty}\fi
\setcounter{page}{4321}
\begin{document}

%%%%%%%%% TITLE
\title{Supplementary Information: Improving Monocular Depth Estimation with
  Global Depth Histogram Matching}

\maketitle
\section{Ablation study on number of SID bins}
\begin{itemize}
  \item Simulation: 70, 140, 210, 280 bins
  \item Captured: 70, 140, 210, 280 bins
\end{itemize}
Compare RMSE and runtime.
\section{Ablation study on effect of reflectance estimation}
\begin{itemize}
  \item Intensity used to initialize histogram, not used to move pixels
  \item Intensity used to initialize histogram and move pixels
  \item Intensity not used to initialize histogram, used to move pixels
  \item Intensity not used to initialize histogram, not used to move pixels
\end{itemize}
Compare RMSE and runtime.

\section{Additional results on NYU Depth v2}
Figures \ref{fig:densedepth_1}--\ref{fig:dorn_3} show additional results
for our method on the NYU Depth v2 dataset when the depth estimate is
initialized with the DenseDepth \cite{Alhashim2018} (Figures
\ref{fig:densedepth_1}--\ref{fig:densedepth_3}), and DORN \cite{Fu2018} (Figures
\ref{fig:dorn_1}--\ref{fig:dorn_3}) monocular depth estimators.
% and MiDaS \cite{MiDaS} (Figures \ref{midas_1} -- \ref{midas_3}).
% network of \cite{Alhashim2019}.
We compare the output of the network $z_0$, the
median-rescaled network output (where the depth map $z_0$ is scaled pixel-wise by a
scalar $\frac{\text{median}(z_{GT})}{\text{median}(z_0)}$, $z_{GT}$ being the
ground truth depth map), the network output matched to the ground truth depth histogram, and the output of
our histogram matching method under a signal-to-background ratio (SBR) of 100.
We use the Y channel of the YCbCr-transformed RGB image as our reflectance map
for both SPAD simulation and histogram matching.
We show absolute difference maps and also give
the root-mean-square error (RMSE) for each example.
\begin{figure}
  \includegraphics[width=\textwidth]{comparison/densedepth_18_comparison.pdf}
  \includegraphics[width=\textwidth]{comparison/densedepth_21_comparison.pdf}
  \includegraphics[width=\textwidth]{comparison/densedepth_25_comparison.pdf}
  \includegraphics[width=\textwidth]{comparison/densedepth_194_comparison.pdf}
  \caption{Results on DenseDepth}
\end{figure}
\begin{figure}
  \includegraphics[width=\textwidth]{comparison/densedepth_285_comparison.pdf}
  \includegraphics[width=\textwidth]{comparison/densedepth_42_comparison.pdf}
  \includegraphics[width=\textwidth]{comparison/densedepth_52_comparison.pdf}
  \includegraphics[width=\textwidth]{comparison/densedepth_53_comparison.pdf}
  \caption{Results on DenseDepth}
\end{figure}
\begin{figure}
  \includegraphics[width=\textwidth]{comparison/densedepth_103_comparison.pdf}
  \includegraphics[width=\textwidth]{comparison/densedepth_224_comparison.pdf}
  \includegraphics[width=\textwidth]{comparison/densedepth_226_comparison.pdf}
  \includegraphics[width=\textwidth]{comparison/densedepth_352_comparison.pdf}
  \caption{Results on DenseDepth}
\end{figure}
% \begin{figure}
%   \includegraphics[width=\textwidth]{comparison/densedepth_362_comparison.pdf}
%   \includegraphics[width=\textwidth]{comparison/densedepth_379_comparison.pdf}
%   \includegraphics[width=\textwidth]{comparison/densedepth_107_comparison.pdf}
%   \includegraphics[width=\textwidth]{comparison/densedepth_187_comparison.pdf}
%   \caption{Results on DenseDepth}
% \end{figure}
% \begin{figure}
%   \includegraphics[width=\textwidth]{comparison/densedepth_140_comparison.pdf}
%   \includegraphics[width=\textwidth]{comparison/densedepth_244_comparison.pdf}
%   \includegraphics[width=\textwidth]{comparison/densedepth_527_comparison.pdf}
%   \includegraphics[width=\textwidth]{comparison/densedepth_529_comparison.pdf}
%   \caption{Results on DenseDepth}
% \end{figure}
% \begin{figure}
%   \includegraphics[width=\textwidth]{comparison/densedepth_529_comparison.pdf}
%   \includegraphics[width=\textwidth]{comparison/densedepth_219_comparison.pdf}
%   \includegraphics[width=\textwidth]{comparison/densedepth_329_comparison.pdf}
%   \includegraphics[width=\textwidth]{comparison/densedepth_346_comparison.pdf}
%   \caption{Results on DenseDepth}
% \end{figure}
% \begin{figure}
%   \includegraphics[width=\textwidth]{comparison/densedepth_616_comparison.pdf}
%   \includegraphics[width=\textwidth]{comparison/densedepth_390_comparison.pdf}
%   \includegraphics[width=\textwidth]{comparison/densedepth_412_comparison.pdf}
%   \includegraphics[width=\textwidth]{comparison/densedepth_420_comparison.pdf}
%   \caption{Results on DenseDepth}
% \end{figure}
% \begin{figure}
%   \includegraphics[width=\textwidth]{comparison/densedepth_428_comparison.pdf}
%   \includegraphics[width=\textwidth]{comparison/densedepth_457_comparison.pdf}
%   \includegraphics[width=\textwidth]{comparison/densedepth_468_comparison.pdf}
%   \includegraphics[width=\textwidth]{comparison/densedepth_477_comparison.pdf}
%   \caption{Results on DenseDepth}
% \end{figure}
%   \includegraphics[height=3cm]{comparison/densedepth_499_comparison.pdf}
%   \includegraphics[height=3cm]{comparison/densedepth_258_comparison.pdf}
%   \includegraphics[height=3cm]{comparison/densedepth_341_comparison.pdf}
\begin{figure}
  \includegraphics[width=\textwidth]{comparison/dorn_8_comparison.pdf}
  \includegraphics[width=\textwidth]{comparison/dorn_14_comparison.pdf}
  \includegraphics[width=\textwidth]{comparison/dorn_15_comparison.pdf}
  \includegraphics[width=\textwidth]{comparison/dorn_23_comparison.pdf}
  \caption{Results on DORN}
\end{figure}
\begin{figure}
  \includegraphics[width=\textwidth]{comparison/dorn_26_comparison.pdf}
  \includegraphics[width=\textwidth]{comparison/dorn_63_comparison.pdf}
  \includegraphics[width=\textwidth]{comparison/dorn_67_comparison.pdf}
  \includegraphics[width=\textwidth]{comparison/dorn_140_comparison.pdf}
  \caption{Results on DORN}
\end{figure}
\begin{figure}
  \includegraphics[width=\textwidth]{comparison/dorn_170_comparison.pdf}
  \includegraphics[width=\textwidth]{comparison/dorn_522_comparison.pdf}
  \includegraphics[width=\textwidth]{comparison/dorn_548_comparison.pdf}
  \includegraphics[width=\textwidth]{comparison/dorn_569_comparison.pdf}
  \caption{Results on DORN}
\end{figure}
% \begin{figure}
%   \includegraphics[width=\textwidth]{comparison/dorn_585_comparison.pdf}
%   \includegraphics[width=\textwidth]{comparison/dorn_633_comparison.pdf}
%   \includegraphics[width=\textwidth]{comparison/dorn_105_comparison.pdf}
%   \includegraphics[width=\textwidth]{comparison/dorn_220_comparison.pdf}
%   \caption{Results on DORN}
% \end{figure}
% \begin{figure}
%   \includegraphics[width=\textwidth]{comparison/dorn_252_comparison.pdf}
%   \includegraphics[width=\textwidth]{comparison/dorn_259_comparison.pdf}
%   \includegraphics[width=\textwidth]{comparison/dorn_178_comparison.pdf}
%   \includegraphics[width=\textwidth]{comparison/dorn_219_comparison.pdf}
%   \caption{Results on DORN}
% \end{figure}
% \begin{figure}
%   \includegraphics[width=\textwidth]{comparison/dorn_280_comparison.pdf}
%   \includegraphics[width=\textwidth]{comparison/dorn_293_comparison.pdf}
%   \includegraphics[width=\textwidth]{comparison/dorn_313_comparison.pdf}
%   \includegraphics[width=\textwidth]{comparison/dorn_413_comparison.pdf}
%   \caption{Results on DORN}
% \end{figure}
% \begin{figure}
  % \includegraphics[width=\textwidth]{comparison/dorn_477_comparison.pdf}
  % \includegraphics[width=\textwidth]{comparison/dorn_502_comparison.pdf}
%  \includegraphics[width=\textwidth]{comparison/dorn_428_comparison.pdf}
%  \includegraphics[width=\textwidth]{comparison/dorn_428_comparison.pdf}
  % \caption{Results on DORN}
% \end{figure}
% Densedepth:
% - Office: 18, 21
% - Bathroom: 25, 194, 285
% - Bookstore: 42
% - Kitchen: 52, 53, 103, 224, 226, 352, 362, 379
% - Office: 107, 187
% - Living Room: 140, 244, 527, 529
% - Dining Room: 219, 329, 346, 616
% - Bedroom: 390, 412, 420, 428, 457*, 468*, 477, 499
% Failure Cases: 258, 338(?), 341, 537 649 (minor)
% 
% DORN:
% - Office: 8, 14, 15, 23, 26
% - Living room: 63, 67, 140, 170, 522, 548, 569, 585, 633
% - Office: 105, 220, 252, 259* (interesting)
% - Kitchen: 178
% - Dining Room: 219
% - Bathroom: 280, 293, 313
% - Bedroom: 413, 477, 502
% 
% Failure Cases: 96, 266(?)



\section{Additional results for hardware prototype}
Figures \ref{fig:midas_captured_1}--\ref{fig:dorn_captured_3} show all the
captured results when the depth estimate is initialized with the MiDaS
\cite{Lasinger:2019} (Figures \ref{fig:midas_captured_1}--\ref{fig:midas_captured_3}),
DenseDepth (Figures \ref{fig:densedepth_captured_1}--\ref{fig:densedepth_captured_3}),
and DORN (Figures \ref{fig:dorn_captured_1}--\ref{fig:dorn_captured_3}). We compare
the output of the network $z_0$, the mean-rescaled network output where the
depth map $z_0$ has been scaled pixel-wise by the scalar
$\frac{\text{median}(h_{target})}{\text{median}(z_0)}$ ($h_{target}$ is the
processed SPAD transient), and the output of our method. As our laser is red, we
use the R channel of the RGB image as our reflectance map. We show absolute
difference maps and also give the root-mean-square-error (RMSE) for each
example.

Black pixels in the ground truth depth correspond to locations where our scanner
was unable to produce an accurate depth estimate (this can occur for a variety
of reasons including dark albedo and surface specularity). These pixels are masked off and not used in the RMSE
calculation, and appear as an absolute difference of 0 in the difference maps.
\begin{figure}
    \centering
    \begin{tabular}{p{5mm}*{4}{>{\centering\arraybackslash}p{1.15in}}c}
      \multirow[t]{5}{=}[-1in]{\rotatebox[origin=rc]{90}{Kitchen Scene}} & Ground Truth & CNN & CNN Mean Rescaled & CNN Histogram Matched & \\
      &
      \includegraphics[height=1.15in, width=1.15in]{captured/midas/8_29_kitchen_scene/gt_z_proj_crop_depth_fig.png}&
      \includegraphics[height=1.15in, width=1.15in]{captured/midas/8_29_kitchen_scene/z_init_depth_fig.png}&
      \includegraphics[height=1.15in, width=1.15in]{captured/midas/8_29_kitchen_scene/z_med_scaled_depth_fig.png}&
      \includegraphics[height=1.15in, width=1.15in]{captured/midas/8_29_kitchen_scene/z_pred_depth_fig.png}&
      \includegraphics[height=1.15in]{captured/midas/8_29_kitchen_scene/depth_colorbar.pdf}\\
%      & RGB &  &  &  & \\
      &
      \includegraphics[height=1.15in, width=1.15in]{captured/midas/8_29_kitchen_scene/rgb_cropped_fig.png}&
      \includegraphics[height=1.15in, width=1.15in]{captured/midas/8_29_kitchen_scene/z_init_diff_fig.png}&
      \includegraphics[height=1.15in, width=1.15in]{captured/midas/8_29_kitchen_scene/z_med_scaled_diff_fig.png}&
      \includegraphics[height=1.15in, width=1.15in]{captured/midas/8_29_kitchen_scene/z_pred_diff_fig.png}&
      \includegraphics[height=1.15in]{captured/midas/8_29_kitchen_scene/diff_colorbar.pdf}\\
      & RGB & RMSE = 1.784 & RMSE = 2.454 & RMSE = 0.350& \\ 

      \rule{0pt}{3ex}  & & & & & \\
      \multirow[t]{3}{=}{\rotatebox[origin=c]{90}{Conference Room 1}}& 
      \includegraphics[height=1.15in, width=1.15in]{captured/midas/8_29_conference_room_scene/gt_z_proj_crop_depth_fig.png}&
      \includegraphics[height=1.15in, width=1.15in]{captured/midas/8_29_conference_room_scene/z_init_depth_fig.png}&
      \includegraphics[height=1.15in, width=1.15in]{captured/midas/8_29_conference_room_scene/z_med_scaled_depth_fig.png}&
      \includegraphics[height=1.15in, width=1.15in]{captured/midas/8_29_conference_room_scene/z_pred_depth_fig.png}&
      \includegraphics[height=1.15in]{captured/midas/8_29_conference_room_scene/depth_colorbar.pdf}\\

      &
      \includegraphics[height=1.15in, width=1.15in]{captured/midas/8_29_conference_room_scene/rgb_cropped_fig.png}&
      \includegraphics[height=1.15in, width=1.15in]{captured/midas/8_29_conference_room_scene/z_init_diff_fig.png}&
      \includegraphics[height=1.15in, width=1.15in]{captured/midas/8_29_conference_room_scene/z_med_scaled_diff_fig.png}&
      \includegraphics[height=1.15in, width=1.15in]{captured/midas/8_29_conference_room_scene/z_pred_diff_fig.png}&
      \includegraphics[height=1.15in]{captured/midas/8_29_conference_room_scene/diff_colorbar.pdf}\\
      & & RMSE = 1.784 & RMSE = 2.454 & RMSE = 0.350& \\ 

      \rule{0pt}{3ex}  & & & & & \\
      \multirow[t]{3}{=}{\rotatebox[origin=c]{90}{Conference Room 2}}&
      \includegraphics[height=1.15in, width=1.15in]{captured/midas/8_30_conference_room2_scene/gt_z_proj_crop_depth_fig.png}&
      \includegraphics[height=1.15in, width=1.15in]{captured/midas/8_30_conference_room2_scene/z_init_depth_fig.png}&
      \includegraphics[height=1.15in, width=1.15in]{captured/midas/8_30_conference_room2_scene/z_med_scaled_depth_fig.png}&
      \includegraphics[height=1.15in, width=1.15in]{captured/midas/8_30_conference_room2_scene/z_pred_depth_fig.png}&
      \includegraphics[height=1.15in]{captured/midas/8_30_conference_room2_scene/depth_colorbar.pdf}\\

      &
      \includegraphics[height=1.15in, width=1.15in]{captured/midas/8_30_conference_room2_scene/rgb_cropped_fig.png}&
      \includegraphics[height=1.15in, width=1.15in]{captured/midas/8_30_conference_room2_scene/z_init_diff_fig.png}&
      \includegraphics[height=1.15in, width=1.15in]{captured/midas/8_30_conference_room2_scene/z_med_scaled_diff_fig.png}&
      \includegraphics[height=1.15in, width=1.15in]{captured/midas/8_30_conference_room2_scene/z_pred_diff_fig.png}&
      \includegraphics[height=1.15in]{captured/midas/8_30_conference_room2_scene/diff_colorbar.pdf}\\
      & & RMSE = 1.784 & RMSE = 2.454 & RMSE = 0.350& \\ 
    \end{tabular}
    \caption{Captured results initialized using the MiDaS CNN.
      Second row shows absolute difference between above estimates and ground truth.}
    \label{fig:midas_captured}
%    \vspace{-1.5em}
\end{figure}
\begin{figure}
    \centering
    \begin{tabular}{p{5mm}*{4}{>{\centering\arraybackslash}p{1.15in}}c}
      \multirow[t]{5}{=}[-1in]{\rotatebox[origin=rc]{90}{Hallway}} & Ground Truth & CNN & CNN Mean Rescaled & CNN Histogram Matched & \\
      &
      \includegraphics[width=1.15in, height=1.15in]{captured/midas/8_30_Hallway/gt_z_proj_crop_depth_fig.png}&
      \includegraphics[width=1.15in, height=1.15in]{captured/midas/8_30_Hallway/z_init_depth_fig.png}&
      \includegraphics[width=1.15in, height=1.15in]{captured/midas/8_30_Hallway/z_med_scaled_depth_fig.png}&
      \includegraphics[width=1.15in, height=1.15in]{captured/midas/8_30_Hallway/z_pred_depth_fig.png}&
      \includegraphics[height=1.15in]{captured/midas/8_30_Hallway/depth_colorbar.pdf}\\
      & & & & & \\

      & 
      \includegraphics[width=1.15in, height=1.15in]{captured/midas/8_30_Hallway/rgb_cropped_fig.png}&
      \includegraphics[width=1.15in, height=1.15in]{captured/midas/8_30_Hallway/z_init_diff_fig.png}&
      \includegraphics[width=1.15in, height=1.15in]{captured/midas/8_30_Hallway/z_med_scaled_diff_fig.png}&
      \includegraphics[width=1.15in, height=1.15in]{captured/midas/8_30_Hallway/z_pred_diff_fig.png}&
      \includegraphics[height=1.15in]{captured/midas/8_30_Hallway/diff_colorbar.pdf}\\
      & RGB & RMSE = 1.784 & RMSE = 2.454 & RMSE = 0.350& \\ 

      \rule{0pt}{3ex}  & & & & & \\
      \multirow[t]{3}{=}{\rotatebox[origin=c]{90}{Poster}}&
      \includegraphics[width=1.15in, height=1.15in]{captured/midas/8_30_poster_scene/gt_z_proj_crop_depth_fig.png}&
      \includegraphics[width=1.15in, height=1.15in]{captured/midas/8_30_poster_scene/z_init_depth_fig.png}&
      \includegraphics[width=1.15in, height=1.15in]{captured/midas/8_30_poster_scene/z_med_scaled_depth_fig.png}&
      \includegraphics[width=1.15in, height=1.15in]{captured/midas/8_30_poster_scene/z_pred_depth_fig.png}&
      \includegraphics[height=1.15in]{captured/midas/8_30_poster_scene/depth_colorbar.pdf}\\

      &
      \includegraphics[width=1.15in, height=1.15in]{captured/midas/8_30_poster_scene/rgb_cropped_fig.png}&
      \includegraphics[width=1.15in, height=1.15in]{captured/midas/8_30_poster_scene/z_init_diff_fig.png}&
      \includegraphics[width=1.15in, height=1.15in]{captured/midas/8_30_poster_scene/z_med_scaled_diff_fig.png}&
      \includegraphics[width=1.15in, height=1.15in]{captured/midas/8_30_poster_scene/z_pred_diff_fig.png}&
      \includegraphics[height=1.15in]{captured/midas/8_30_poster_scene/diff_colorbar.pdf}\\
      & & RMSE = 1.784 & RMSE = 2.454 & RMSE = 0.350& \\ 

      \rule{0pt}{3ex}  & & & & & \\
      \multirow[t]{3}{=}{\rotatebox[origin=c]{90}{Bookshelf}}&
      \includegraphics[width=1.15in, height=1.15in]{captured/midas/8_30_small_lab_scene/gt_z_proj_crop_depth_fig.png}&
      \includegraphics[width=1.15in, height=1.15in]{captured/midas/8_30_small_lab_scene/z_init_depth_fig.png}&
      \includegraphics[width=1.15in, height=1.15in]{captured/midas/8_30_small_lab_scene/z_med_scaled_depth_fig.png}&
      \includegraphics[width=1.15in, height=1.15in]{captured/midas/8_30_small_lab_scene/z_pred_depth_fig.png}&
      \includegraphics[height=1.15in]{captured/midas/8_30_small_lab_scene/depth_colorbar.pdf}\\

      &
      \includegraphics[width=1.15in, height=1.15in]{captured/midas/8_30_small_lab_scene/rgb_cropped_fig.png}&
      \includegraphics[width=1.15in, height=1.15in]{captured/midas/8_30_small_lab_scene/z_init_diff_fig.png}&
      \includegraphics[width=1.15in, height=1.15in]{captured/midas/8_30_small_lab_scene/z_med_scaled_diff_fig.png}&
      \includegraphics[width=1.15in, height=1.15in]{captured/midas/8_30_small_lab_scene/z_pred_diff_fig.png}&
      \includegraphics[height=1.15in]{captured/midas/8_30_small_lab_scene/diff_colorbar.pdf}\\
      & & RMSE = 1.784 & RMSE = 2.454 & RMSE = 0.350\\ 
                                     
%      (a)&(b)&(c)&(d)&(e)\\
    \end{tabular}
    \caption{Captured results initialized using the MiDaS CNN.
      Second row shows absolute difference between above estimates and ground truth.}
    \label{fig:midas_captured}
%    \vspace{-1.5em}
\end{figure}
 
% %% MiDaS outdoor scene
\begin{figure}
    \centering
    \begin{tabular}{p{5mm}*{4}{>{\centering\arraybackslash}p{1.15in}}c}
      \multirow[t]{5}{=}[-1in]{\rotatebox[origin=rc]{90}{Outdoor Scene}} & Ground Truth & CNN & CNN Mean Rescaled & CNN Histogram Matched & \\
      &
      \includegraphics[height=1.15in, width=1.15in]{captured/midas/8_31_outdoor3/gt_z_proj_crop_depth_fig.png}&
      \includegraphics[height=1.15in, width=1.15in]{captured/midas/8_31_outdoor3/z_init_depth_fig.png}&
      \includegraphics[height=1.15in, width=1.15in]{captured/midas/8_31_outdoor3/z_med_scaled_depth_fig.png}&
      \includegraphics[height=1.15in, width=1.15in]{captured/midas/8_31_outdoor3/z_pred_depth_fig.png}&
      \includegraphics[height=1.15in]{captured/midas/8_31_outdoor3/depth_colorbar.pdf}\\
%      & RGB &  &  &  & \\
      &
      \includegraphics[height=1.15in, width=1.15in]{captured/midas/8_31_outdoor3/rgb_cropped_fig.png}&
      \includegraphics[height=1.15in, width=1.15in]{captured/midas/8_31_outdoor3/z_init_diff_fig.png}&
      \includegraphics[height=1.15in, width=1.15in]{captured/midas/8_31_outdoor3/z_med_scaled_diff_fig.png}&
      \includegraphics[height=1.15in, width=1.15in]{captured/midas/8_31_outdoor3/z_pred_diff_fig.png}&
      \includegraphics[height=1.15in]{captured/midas/8_31_outdoor3/diff_colorbar.pdf}\\
      & RGB & RMSE = 1.784 & RMSE = 2.454 & RMSE = 0.350& \\ 
    \end{tabular}
    \caption{Captured results initialized using the MiDaS CNN on an outdoor scene.
      Second row shows absolute difference between above estimates and ground truth.}
    \label{fig:midas_outdoor_captured}
%    \vspace{-1.5em}
\end{figure}


\begin{figure}
    \centering
    \begin{tabular}{p{5mm}*{4}{>{\centering\arraybackslash}p{1.15in}}c}
      \multirow[t]{5}{=}[-1in]{\rotatebox[origin=rc]{90}{Kitchen Scene}} & Ground Truth & CNN & CNN Mean Rescaled & CNN Histogram Matched & \\
      &
      \includegraphics[height=1.15in, width=1.15in]{captured/densedepth/8_29_kitchen_scene/gt_z_proj_crop_depth_fig.png}&
      \includegraphics[height=1.15in, width=1.15in]{captured/densedepth/8_29_kitchen_scene/z_init_depth_fig.png}&
      \includegraphics[height=1.15in, width=1.15in]{captured/densedepth/8_29_kitchen_scene/z_med_scaled_depth_fig.png}&
      \includegraphics[height=1.15in, width=1.15in]{captured/densedepth/8_29_kitchen_scene/z_pred_depth_fig.png}&
      \includegraphics[height=1.15in]{captured/densedepth/8_29_kitchen_scene/depth_colorbar.pdf}\\
%      & RGB &  &  &  & \\
      &
      \includegraphics[height=1.15in, width=1.15in]{captured/densedepth/8_29_kitchen_scene/rgb_cropped_fig.png}&
      \includegraphics[height=1.15in, width=1.15in]{captured/densedepth/8_29_kitchen_scene/z_init_diff_fig.png}&
      \includegraphics[height=1.15in, width=1.15in]{captured/densedepth/8_29_kitchen_scene/z_med_scaled_diff_fig.png}&
      \includegraphics[height=1.15in, width=1.15in]{captured/densedepth/8_29_kitchen_scene/z_pred_diff_fig.png}&
      \includegraphics[height=1.15in]{captured/densedepth/8_29_kitchen_scene/diff_colorbar.pdf}\\
      & RGB & RMSE = 1.784 & RMSE = 2.454 & RMSE = 0.350& \\ 

      \rule{0pt}{3ex}  & & & & & \\
      \multirow[t]{3}{=}{\rotatebox[origin=c]{90}{Conference Room 1}}& 
      \includegraphics[height=1.15in, width=1.15in]{captured/densedepth/8_29_conference_room_scene/gt_z_proj_crop_depth_fig.png}&
      \includegraphics[height=1.15in, width=1.15in]{captured/densedepth/8_29_conference_room_scene/z_init_depth_fig.png}&
      \includegraphics[height=1.15in, width=1.15in]{captured/densedepth/8_29_conference_room_scene/z_med_scaled_depth_fig.png}&
      \includegraphics[height=1.15in, width=1.15in]{captured/densedepth/8_29_conference_room_scene/z_pred_depth_fig.png}&
      \includegraphics[height=1.15in]{captured/densedepth/8_29_conference_room_scene/depth_colorbar.pdf}\\

      &
      \includegraphics[height=1.15in, width=1.15in]{captured/densedepth/8_29_conference_room_scene/rgb_cropped_fig.png}&
      \includegraphics[height=1.15in, width=1.15in]{captured/densedepth/8_29_conference_room_scene/z_init_diff_fig.png}&
      \includegraphics[height=1.15in, width=1.15in]{captured/densedepth/8_29_conference_room_scene/z_med_scaled_diff_fig.png}&
      \includegraphics[height=1.15in, width=1.15in]{captured/densedepth/8_29_conference_room_scene/z_pred_diff_fig.png}&
      \includegraphics[height=1.15in]{captured/densedepth/8_29_conference_room_scene/diff_colorbar.pdf}\\
      & & RMSE = 1.784 & RMSE = 2.454 & RMSE = 0.350& \\ 

      \rule{0pt}{3ex}  & & & & & \\
      \multirow[t]{3}{=}{\rotatebox[origin=c]{90}{Conference Room 2}}&
      \includegraphics[height=1.15in, width=1.15in]{captured/densedepth/8_30_conference_room2_scene/gt_z_proj_crop_depth_fig.png}&
      \includegraphics[height=1.15in, width=1.15in]{captured/densedepth/8_30_conference_room2_scene/z_init_depth_fig.png}&
      \includegraphics[height=1.15in, width=1.15in]{captured/densedepth/8_30_conference_room2_scene/z_med_scaled_depth_fig.png}&
      \includegraphics[height=1.15in, width=1.15in]{captured/densedepth/8_30_conference_room2_scene/z_pred_depth_fig.png}&
      \includegraphics[height=1.15in]{captured/densedepth/8_30_conference_room2_scene/depth_colorbar.pdf}\\

      &
      \includegraphics[height=1.15in, width=1.15in]{captured/densedepth/8_30_conference_room2_scene/rgb_cropped_fig.png}&
      \includegraphics[height=1.15in, width=1.15in]{captured/densedepth/8_30_conference_room2_scene/z_init_diff_fig.png}&
      \includegraphics[height=1.15in, width=1.15in]{captured/densedepth/8_30_conference_room2_scene/z_med_scaled_diff_fig.png}&
      \includegraphics[height=1.15in, width=1.15in]{captured/densedepth/8_30_conference_room2_scene/z_pred_diff_fig.png}&
      \includegraphics[height=1.15in]{captured/densedepth/8_30_conference_room2_scene/diff_colorbar.pdf}\\
      & & RMSE = 1.784 & RMSE = 2.454 & RMSE = 0.350& \\ 
    \end{tabular}
    \caption{Captured results initialized using the DenseDepth CNN.
      Second row shows absolute difference between above estimates and ground truth.}
    \label{fig:densedepth_captured}
%    \vspace{-1.5em}
\end{figure}
\begin{figure}
    \centering
    \begin{tabular}{p{5mm}*{4}{>{\centering\arraybackslash}p{1.15in}}c}
      \multirow[t]{5}{=}[-1in]{\rotatebox[origin=rc]{90}{Hallway}} & Ground Truth & CNN & CNN Mean Rescaled & CNN Histogram Matched & \\
      &
      \includegraphics[width=1.15in, height=1.15in]{captured/densedepth/8_30_Hallway/gt_z_proj_crop_depth_fig.png}&
      \includegraphics[width=1.15in, height=1.15in]{captured/densedepth/8_30_Hallway/z_init_depth_fig.png}&
      \includegraphics[width=1.15in, height=1.15in]{captured/densedepth/8_30_Hallway/z_med_scaled_depth_fig.png}&
      \includegraphics[width=1.15in, height=1.15in]{captured/densedepth/8_30_Hallway/z_pred_depth_fig.png}&
      \includegraphics[height=1.15in]{captured/densedepth/8_30_Hallway/depth_colorbar.pdf}\\
      & & & & & \\

      & 
      \includegraphics[width=1.15in, height=1.15in]{captured/densedepth/8_30_Hallway/rgb_cropped_fig.png}&
      \includegraphics[width=1.15in, height=1.15in]{captured/densedepth/8_30_Hallway/z_init_diff_fig.png}&
      \includegraphics[width=1.15in, height=1.15in]{captured/densedepth/8_30_Hallway/z_med_scaled_diff_fig.png}&
      \includegraphics[width=1.15in, height=1.15in]{captured/densedepth/8_30_Hallway/z_pred_diff_fig.png}&
      \includegraphics[height=1.15in]{captured/densedepth/8_30_Hallway/diff_colorbar.pdf}\\
      & RGB & RMSE = 1.784 & RMSE = 2.454 & RMSE = 0.350& \\ 

      \rule{0pt}{3ex}  & & & & & \\
      \multirow[t]{3}{=}{\rotatebox[origin=c]{90}{Poster}}&
      \includegraphics[width=1.15in, height=1.15in]{captured/densedepth/8_30_poster_scene/gt_z_proj_crop_depth_fig.png}&
      \includegraphics[width=1.15in, height=1.15in]{captured/densedepth/8_30_poster_scene/z_init_depth_fig.png}&
      \includegraphics[width=1.15in, height=1.15in]{captured/densedepth/8_30_poster_scene/z_med_scaled_depth_fig.png}&
      \includegraphics[width=1.15in, height=1.15in]{captured/densedepth/8_30_poster_scene/z_pred_depth_fig.png}&
      \includegraphics[height=1.15in]{captured/densedepth/8_30_poster_scene/depth_colorbar.pdf}\\

      &
      \includegraphics[width=1.15in, height=1.15in]{captured/densedepth/8_30_poster_scene/rgb_cropped_fig.png}&
      \includegraphics[width=1.15in, height=1.15in]{captured/densedepth/8_30_poster_scene/z_init_diff_fig.png}&
      \includegraphics[width=1.15in, height=1.15in]{captured/densedepth/8_30_poster_scene/z_med_scaled_diff_fig.png}&
      \includegraphics[width=1.15in, height=1.15in]{captured/densedepth/8_30_poster_scene/z_pred_diff_fig.png}&
      \includegraphics[height=1.15in]{captured/densedepth/8_30_poster_scene/diff_colorbar.pdf}\\
      & & RMSE = 1.784 & RMSE = 2.454 & RMSE = 0.350& \\ 

      \rule{0pt}{3ex}  & & & & & \\
      \multirow[t]{3}{=}{\rotatebox[origin=c]{90}{Bookshelf}}&
      \includegraphics[width=1.15in, height=1.15in]{captured/densedepth/8_30_small_lab_scene/gt_z_proj_crop_depth_fig.png}&
      \includegraphics[width=1.15in, height=1.15in]{captured/densedepth/8_30_small_lab_scene/z_init_depth_fig.png}&
      \includegraphics[width=1.15in, height=1.15in]{captured/densedepth/8_30_small_lab_scene/z_med_scaled_depth_fig.png}&
      \includegraphics[width=1.15in, height=1.15in]{captured/densedepth/8_30_small_lab_scene/z_pred_depth_fig.png}&
      \includegraphics[height=1.15in]{captured/densedepth/8_30_small_lab_scene/depth_colorbar.pdf}\\

      &
      \includegraphics[width=1.15in, height=1.15in]{captured/densedepth/8_30_small_lab_scene/rgb_cropped_fig.png}&
      \includegraphics[width=1.15in, height=1.15in]{captured/densedepth/8_30_small_lab_scene/z_init_diff_fig.png}&
      \includegraphics[width=1.15in, height=1.15in]{captured/densedepth/8_30_small_lab_scene/z_med_scaled_diff_fig.png}&
      \includegraphics[width=1.15in, height=1.15in]{captured/densedepth/8_30_small_lab_scene/z_pred_diff_fig.png}&
      \includegraphics[height=1.15in]{captured/densedepth/8_30_small_lab_scene/diff_colorbar.pdf}\\
      & & RMSE = 1.784 & RMSE = 2.454 & RMSE = 0.350\\ 
                                     
%      (a)&(b)&(c)&(d)&(e)\\
    \end{tabular}
    \caption{Captured results initialized using the DenseDepth CNN.
      Second row shows absolute difference between above estimates and ground truth.}
    \label{fig:densedepth_captured}
%    \vspace{-1.5em}
\end{figure}
%% DenseDepth outdoor scene
\begin{figure}
    \centering
    \begin{tabular}{p{5mm}*{4}{>{\centering\arraybackslash}p{1.15in}}c}
      \multirow[t]{5}{=}[-1in]{\rotatebox[origin=rc]{90}{Outdoor Scene}} & Ground Truth & CNN & CNN Mean Rescaled & CNN Histogram Matched & \\
      &
      \includegraphics[height=1.15in, width=1.15in]{captured/densedepth/8_31_outdoor3/gt_z_proj_crop_depth_fig.png}&
      \includegraphics[height=1.15in, width=1.15in]{captured/densedepth/8_31_outdoor3/z_init_depth_fig.png}&
      \includegraphics[height=1.15in, width=1.15in]{captured/densedepth/8_31_outdoor3/z_med_scaled_depth_fig.png}&
      \includegraphics[height=1.15in, width=1.15in]{captured/densedepth/8_31_outdoor3/z_pred_depth_fig.png}&
      \includegraphics[height=1.15in]{captured/densedepth/8_31_outdoor3/depth_colorbar.pdf}\\
%      & RGB &  &  &  & \\
      &
      \includegraphics[height=1.15in, width=1.15in]{captured/densedepth/8_31_outdoor3/rgb_cropped_fig.png}&
      \includegraphics[height=1.15in, width=1.15in]{captured/densedepth/8_31_outdoor3/z_init_diff_fig.png}&
      \includegraphics[height=1.15in, width=1.15in]{captured/densedepth/8_31_outdoor3/z_med_scaled_diff_fig.png}&
      \includegraphics[height=1.15in, width=1.15in]{captured/densedepth/8_31_outdoor3/z_pred_diff_fig.png}&
      \includegraphics[height=1.15in]{captured/densedepth/8_31_outdoor3/diff_colorbar.pdf}\\
      & RGB & RMSE = 1.784 & RMSE = 2.454 & RMSE = 0.350& \\ 
    \end{tabular}
    \caption{Captured results initialized using the DenseDepth CNN on an outdoor scene.
      Second row shows absolute difference between above estimates and ground truth.}
    \label{fig:densedepth_outdoor_captured}
%    \vspace{-1.5em}
\end{figure}


% %%%%%%%%%%%%%%%%%%%%%%%%%%%%%%%%%%%%%%%
 
\begin{figure}
    \centering
    \begin{tabular}{p{5mm}*{4}{>{\centering\arraybackslash}p{1.15in}}c}
      \multirow[t]{5}{=}[-1in]{\rotatebox[origin=rc]{90}{Kitchen Scene}} & Ground Truth & CNN & CNN Mean Rescaled & CNN Histogram Matched & \\
      &
      \includegraphics[height=1.15in, width=1.15in]{captured/dorn/8_29_kitchen_scene/gt_z_proj_crop_depth_fig.png}&
      \includegraphics[height=1.15in, width=1.15in]{captured/dorn/8_29_kitchen_scene/z_init_depth_fig.png}&
      \includegraphics[height=1.15in, width=1.15in]{captured/dorn/8_29_kitchen_scene/z_med_scaled_depth_fig.png}&
      \includegraphics[height=1.15in, width=1.15in]{captured/dorn/8_29_kitchen_scene/z_pred_depth_fig.png}&
      \includegraphics[height=1.15in]{captured/dorn/8_29_kitchen_scene/depth_colorbar.pdf}\\
%      & RGB &  &  &  & \\
      &
      \includegraphics[height=1.15in, width=1.15in]{captured/dorn/8_29_kitchen_scene/rgb_cropped_fig.png}&
      \includegraphics[height=1.15in, width=1.15in]{captured/dorn/8_29_kitchen_scene/z_init_diff_fig.png}&
      \includegraphics[height=1.15in, width=1.15in]{captured/dorn/8_29_kitchen_scene/z_med_scaled_diff_fig.png}&
      \includegraphics[height=1.15in, width=1.15in]{captured/dorn/8_29_kitchen_scene/z_pred_diff_fig.png}&
      \includegraphics[height=1.15in]{captured/dorn/8_29_kitchen_scene/diff_colorbar.pdf}\\
      & RGB & RMSE = 1.784 & RMSE = 2.454 & RMSE = 0.350& \\ 

      \rule{0pt}{3ex}  & & & & & \\
      \multirow[t]{3}{=}{\rotatebox[origin=c]{90}{Conference Room 1}}& 
      \includegraphics[height=1.15in, width=1.15in]{captured/dorn/8_29_conference_room_scene/gt_z_proj_crop_depth_fig.png}&
      \includegraphics[height=1.15in, width=1.15in]{captured/dorn/8_29_conference_room_scene/z_init_depth_fig.png}&
      \includegraphics[height=1.15in, width=1.15in]{captured/dorn/8_29_conference_room_scene/z_med_scaled_depth_fig.png}&
      \includegraphics[height=1.15in, width=1.15in]{captured/dorn/8_29_conference_room_scene/z_pred_depth_fig.png}&
      \includegraphics[height=1.15in]{captured/dorn/8_29_conference_room_scene/depth_colorbar.pdf}\\

      &
      \includegraphics[height=1.15in, width=1.15in]{captured/dorn/8_29_conference_room_scene/rgb_cropped_fig.png}&
      \includegraphics[height=1.15in, width=1.15in]{captured/dorn/8_29_conference_room_scene/z_init_diff_fig.png}&
      \includegraphics[height=1.15in, width=1.15in]{captured/dorn/8_29_conference_room_scene/z_med_scaled_diff_fig.png}&
      \includegraphics[height=1.15in, width=1.15in]{captured/dorn/8_29_conference_room_scene/z_pred_diff_fig.png}&
      \includegraphics[height=1.15in]{captured/dorn/8_29_conference_room_scene/diff_colorbar.pdf}\\
      & & RMSE = 1.784 & RMSE = 2.454 & RMSE = 0.350& \\ 

      \rule{0pt}{3ex}  & & & & & \\
      \multirow[t]{3}{=}{\rotatebox[origin=c]{90}{Conference Room 2}}&
      \includegraphics[height=1.15in, width=1.15in]{captured/dorn/8_30_conference_room2_scene/gt_z_proj_crop_depth_fig.png}&
      \includegraphics[height=1.15in, width=1.15in]{captured/dorn/8_30_conference_room2_scene/z_init_depth_fig.png}&
      \includegraphics[height=1.15in, width=1.15in]{captured/dorn/8_30_conference_room2_scene/z_med_scaled_depth_fig.png}&
      \includegraphics[height=1.15in, width=1.15in]{captured/dorn/8_30_conference_room2_scene/z_pred_depth_fig.png}&
      \includegraphics[height=1.15in]{captured/dorn/8_30_conference_room2_scene/depth_colorbar.pdf}\\

      &
      \includegraphics[height=1.15in, width=1.15in]{captured/dorn/8_30_conference_room2_scene/rgb_cropped_fig.png}&
      \includegraphics[height=1.15in, width=1.15in]{captured/dorn/8_30_conference_room2_scene/z_init_diff_fig.png}&
      \includegraphics[height=1.15in, width=1.15in]{captured/dorn/8_30_conference_room2_scene/z_med_scaled_diff_fig.png}&
      \includegraphics[height=1.15in, width=1.15in]{captured/dorn/8_30_conference_room2_scene/z_pred_diff_fig.png}&
      \includegraphics[height=1.15in]{captured/dorn/8_30_conference_room2_scene/diff_colorbar.pdf}\\
      & & RMSE = 1.784 & RMSE = 2.454 & RMSE = 0.350& \\ 
    \end{tabular}
    \caption{Captured results initialized using the DORN CNN.
      Second row shows absolute difference between above estimates and ground truth.}
    \label{fig:dorn_captured_1}
%    \vspace{-1.5em}
\end{figure}

\begin{figure}
    \centering
    \begin{tabular}{p{5mm}*{4}{>{\centering\arraybackslash}p{1.15in}}c}
      \multirow[t]{5}{=}[-1in]{\rotatebox[origin=rc]{90}{Hallway}} & Ground Truth & CNN & CNN Mean Rescaled & CNN Histogram Matched & \\
      &
      \includegraphics[width=1.15in, height=1.15in]{captured/dorn/8_30_Hallway/gt_z_proj_crop_depth_fig.png}&
      \includegraphics[width=1.15in, height=1.15in]{captured/dorn/8_30_Hallway/z_init_depth_fig.png}&
      \includegraphics[width=1.15in, height=1.15in]{captured/dorn/8_30_Hallway/z_med_scaled_depth_fig.png}&
      \includegraphics[width=1.15in, height=1.15in]{captured/dorn/8_30_Hallway/z_pred_depth_fig.png}&
      \includegraphics[height=1.15in]{captured/dorn/8_30_Hallway/depth_colorbar.pdf}\\
      & & & & & \\

      & 
      \includegraphics[width=1.15in, height=1.15in]{captured/dorn/8_30_Hallway/rgb_cropped_fig.png}&
      \includegraphics[width=1.15in, height=1.15in]{captured/dorn/8_30_Hallway/z_init_diff_fig.png}&
      \includegraphics[width=1.15in, height=1.15in]{captured/dorn/8_30_Hallway/z_med_scaled_diff_fig.png}&
      \includegraphics[width=1.15in, height=1.15in]{captured/dorn/8_30_Hallway/z_pred_diff_fig.png}&
      \includegraphics[height=1.15in]{captured/dorn/8_30_Hallway/diff_colorbar.pdf}\\
      & RGB & RMSE = 1.784 & RMSE = 2.454 & RMSE = 0.350& \\ 

      \rule{0pt}{3ex}  & & & & & \\
      \multirow[t]{3}{=}{\rotatebox[origin=c]{90}{Poster}}&
      \includegraphics[width=1.15in, height=1.15in]{captured/dorn/8_30_poster_scene/gt_z_proj_crop_depth_fig.png}&
      \includegraphics[width=1.15in, height=1.15in]{captured/dorn/8_30_poster_scene/z_init_depth_fig.png}&
      \includegraphics[width=1.15in, height=1.15in]{captured/dorn/8_30_poster_scene/z_med_scaled_depth_fig.png}&
      \includegraphics[width=1.15in, height=1.15in]{captured/dorn/8_30_poster_scene/z_pred_depth_fig.png}&
      \includegraphics[height=1.15in]{captured/dorn/8_30_poster_scene/depth_colorbar.pdf}\\

      &
      \includegraphics[width=1.15in, height=1.15in]{captured/dorn/8_30_poster_scene/rgb_cropped_fig.png}&
      \includegraphics[width=1.15in, height=1.15in]{captured/dorn/8_30_poster_scene/z_init_diff_fig.png}&
      \includegraphics[width=1.15in, height=1.15in]{captured/dorn/8_30_poster_scene/z_med_scaled_diff_fig.png}&
      \includegraphics[width=1.15in, height=1.15in]{captured/dorn/8_30_poster_scene/z_pred_diff_fig.png}&
      \includegraphics[height=1.15in]{captured/dorn/8_30_poster_scene/diff_colorbar.pdf}\\
      & & RMSE = 1.784 & RMSE = 2.454 & RMSE = 0.350& \\ 

      \rule{0pt}{3ex}  & & & & & \\
      \multirow[t]{3}{=}{\rotatebox[origin=c]{90}{Bookshelf}}&
      \includegraphics[width=1.15in, height=1.15in]{captured/dorn/8_30_small_lab_scene/gt_z_proj_crop_depth_fig.png}&
      \includegraphics[width=1.15in, height=1.15in]{captured/dorn/8_30_small_lab_scene/z_init_depth_fig.png}&
      \includegraphics[width=1.15in, height=1.15in]{captured/dorn/8_30_small_lab_scene/z_med_scaled_depth_fig.png}&
      \includegraphics[width=1.15in, height=1.15in]{captured/dorn/8_30_small_lab_scene/z_pred_depth_fig.png}&
      \includegraphics[height=1.15in]{captured/dorn/8_30_small_lab_scene/depth_colorbar.pdf}\\

      &
      \includegraphics[width=1.15in, height=1.15in]{captured/dorn/8_30_small_lab_scene/rgb_cropped_fig.png}&
      \includegraphics[width=1.15in, height=1.15in]{captured/dorn/8_30_small_lab_scene/z_init_diff_fig.png}&
      \includegraphics[width=1.15in, height=1.15in]{captured/dorn/8_30_small_lab_scene/z_med_scaled_diff_fig.png}&
      \includegraphics[width=1.15in, height=1.15in]{captured/dorn/8_30_small_lab_scene/z_pred_diff_fig.png}&
      \includegraphics[height=1.15in]{captured/dorn/8_30_small_lab_scene/diff_colorbar.pdf}\\
      & & RMSE = 1.784 & RMSE = 2.454 & RMSE = 0.350\\ 
                                     
%      (a)&(b)&(c)&(d)&(e)\\
    \end{tabular}
    \caption{Captured results initialized using the DORN CNN.
      Second row shows absolute difference between above estimates and ground truth.}
    \label{fig:dorn_captured_2}
%    \vspace{-1.5em}
\end{figure}
% %% DORN outdoor scene
% \begin{figure}
%     \centering
%     \begin{tabular}{p{5mm}*{4}{>{\centering\arraybackslash}p{1.15in}}c}
%       \multirow[t]{5}{=}[-1in]{\rotatebox[origin=rc]{90}{Outdoor Scene}} & Ground Truth & CNN & CNN Mean Rescaled & CNN Histogram Matched & \\
%       &
%       \includegraphics[height=1.15in, width=1.15in]{captured/dorn/8_31_outdoor3/gt_z_proj_crop_depth_fig.png}&
%       \includegraphics[height=1.15in, width=1.15in]{captured/dorn/8_31_outdoor3/z_init_depth_fig.png}&
%       \includegraphics[height=1.15in, width=1.15in]{captured/dorn/8_31_outdoor3/z_med_scaled_depth_fig.png}&
%       \includegraphics[height=1.15in, width=1.15in]{captured/dorn/8_31_outdoor3/z_pred_depth_fig.png}&
%       \includegraphics[height=1.15in]{captured/dorn/8_31_outdoor3/depth_colorbar.pdf}\\
% %      & RGB &  &  &  & \\
%       &
%       \includegraphics[height=1.15in, width=1.15in]{captured/dorn/8_31_outdoor3/rgb_cropped_fig.png}&
%       \includegraphics[height=1.15in, width=1.15in]{captured/dorn/8_31_outdoor3/z_init_diff_fig.png}&
%       \includegraphics[height=1.15in, width=1.15in]{captured/dorn/8_31_outdoor3/z_med_scaled_diff_fig.png}&
%       \includegraphics[height=1.15in, width=1.15in]{captured/dorn/8_31_outdoor3/z_pred_diff_fig.png}&
%       \includegraphics[height=1.15in]{captured/dorn/8_31_outdoor3/diff_colorbar.pdf}\\
%       & RGB & RMSE = 1.784 & RMSE = 2.454 & RMSE = 0.350& \\ 
%     \end{tabular}
%     \caption{Captured results initialized using the DORN CNN on an outdoor scene.
%       Second row shows absolute difference between above estimates and ground truth.}
%     \label{fig:dorn_outdoor_captured}
% %    \vspace{-1.5em}
% \end{figure}

\begin{figure}
    \centering
    \begin{tabular}{p{5mm}*{4}{>{\centering\arraybackslash}p{1.15in}}c}
      \multirow[t]{5}{=}[-1in]{\rotatebox[origin=rc]{90}{Outdoor Scene}} & Ground Truth & CNN & CNN Mean Rescaled & CNN Histogram Matched & \\
      &
      \includegraphics[height=1.15in, width=1.15in]{captured/dorn/8_31_outdoor3/gt_z_proj_crop_depth_fig.png}&
      \includegraphics[height=1.15in, width=1.15in]{captured/dorn/8_31_outdoor3/z_init_depth_fig.png}&
      \includegraphics[height=1.15in, width=1.15in]{captured/dorn/8_31_outdoor3/z_med_scaled_depth_fig.png}&
      \includegraphics[height=1.15in, width=1.15in]{captured/dorn/8_31_outdoor3/z_pred_depth_fig.png}&
      \includegraphics[height=1.15in]{captured/dorn/8_31_outdoor3/depth_colorbar.pdf}\\
%      & RGB &  &  &  & \\
      &
      \includegraphics[height=1.15in, width=1.15in]{captured/dorn/8_31_outdoor3/rgb_cropped_fig.png}&
      \includegraphics[height=1.15in, width=1.15in]{captured/dorn/8_31_outdoor3/z_init_diff_fig.png}&
      \includegraphics[height=1.15in, width=1.15in]{captured/dorn/8_31_outdoor3/z_med_scaled_diff_fig.png}&
      \includegraphics[height=1.15in, width=1.15in]{captured/dorn/8_31_outdoor3/z_pred_diff_fig.png}&
      \includegraphics[height=1.15in]{captured/dorn/8_31_outdoor3/diff_colorbar.pdf}\\
      & RGB & RMSE = 1.784 & RMSE = 2.454 & RMSE = 0.350& \\ 
    \end{tabular}
    \caption{Captured results initialized using the DenseDepth CNN on an outdoor scene.
      Second row shows absolute difference between above estimates and ground truth.}
    \label{fig:dorn_outdoor_captured}
%    \vspace{-1.5em}
\end{figure}



{\small
\bibliographystyle{ieee_fullname}
\bibliography{references}
}

\end{document}
